\chapter{研究目标}

针对新产品导入(NPI)过程的复杂性特征和传统项目管理方法的局限性,本文的研究目标是:

\begin{enumerate}
  \item \textbf{分析 NPI 过程的复杂系统特征}:

  从复杂系统理论的角度,深入剖析 NPI 过程中各组成部分的相互作用、非线性特征和动态演化机制,明确其作为复杂系统的内在特征和表现形式。

  \item \textbf{构建 NPI 过程的网络模型}:

  利用网络科学的方法,将 NPI 过程中的各关键要素(如部门、任务、信息流等)抽象为网络中的节点和边,建立反映 NPI 过程结构和动态特性的网络模型。

  \item \textbf{应用网络分析方法优化 NPI 过程}:

  采用网络分析的工具和算法,识别 NPI 网络中的关键节点、关键路径和潜在瓶颈,评估网络的整体效率和稳健性。

  \item \textbf{提出基于网络分析的 NPI 优化策略}:

  根据分析结果,提出针对性的方法和策略,以优化 NPI 过程中的协作效率、信息共享和资源配置,提升新产品导入的成功率和市场现。

  \item \textbf{验证优化策略的有效性}:

  通过案例研究或模拟实验,验证所提出的优化策略在实际应用中的有效性和可行性,为企业实践提供指导。

\end{enumerate}

通过实现上述研究目标,期望为 NPI 过程的管理提供新的理论视角和方法工具,丰富复杂系统和网络分析在工程管理领域的应用研究,同时为企业提升新产品导入效率和竞争力提供实用的解决方案。