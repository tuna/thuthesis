% !TeX root = ../thuthesis-example.tex
\chapter{国内外研究现状}

\section{开放式创新概述}

开放式创新是指企业在创新过程中不仅依赖内部资源,还广泛利用外部资源,通过合作、技术引进和用户参与等方式提升创新能力。
这一概念由Henry Chesbrough在2003年提出,并迅速成为学术界和企业界关注的热点\citep{chesbrough_beyond_2006}.
 开放式创新的核心在于通过开放的方式获取外部知识和技术,从而加速创新过程,降低成本并提高创新成功率。

\section{国外研究现状}

在国外,开放式创新的研究已经非常广泛,涵盖了多个行业和应用场景。
例如,Chesbrough和Crowther(2006)探讨了高科技行业之外的早期开放式创新采用者,
揭示了开放式创新在不同行业中的普遍性\citep{chesbrough_beyond_2006}. Dahlander和Gann(2010)则研究了开放式创新的广度和深度,
讨论了不同企业在开放式创新中的实践和挑战\citep{dahlander_how_2010}.

关于苹果公司,Davidson(2023)评估了苹果公司在支持差异化创新举措方面的价值链收购策略,
揭示了苹果通过开放式创新获取外部技术和资源的过程\citep{davidson_assessing_2023}. 
此外,Gehani(2016)研究了从身份识别到创新能力的企业品牌价值转变,以苹果公司为例,
说明了开放式创新如何增强企业的创新能力和市场竞争力\citep{gehani2016corporate}.

\section{国内研究现状}

在国内,开放式创新的研究起步较晚,但近年来也取得了一些进展。例如,唐兴通和王崇锋(2022)翻译的《开放式创新》
一书详细介绍了开放式创新的理论框架和实践案例,为国内企业提供了理论指导和实践参考\citep{chesbrough2022}. 此外,
复旦大学出版社出版的《开放式创新:创新方法论之新语境》一书则进一步探讨了开放式创新在中小企业和低科技企业中的应用,
扩展了开放式创新的研究范围\citep{chesbrough2016}.

\section{Apple Watch 开发中的开放式创新研究现状}

Apple Watch作为苹果公司推出的首款智能手表,自2015年发布以来一直受到广泛关注。其成功的背后离不开开放式创新的应用。
Apple通过与外部医疗技术公司合作,引入先进的健康监测技术,并通过开放平台鼓励第三方开发者为Apple Watch开发应用,
从而丰富其生态系统\citep{apple2024}.

然而,目前关于Apple Watch开发过程中开放式创新的系统性研究相对较少。现有研究主要集中在苹果公司的整体创新策略上,而缺乏针对具体产品如Apple Watch的深入分析。因此,本研究将系统探讨Apple Watch开发过程中的开放式创新应用,填补这一领域的研究空白。

\section{研究不足与空白}

虽然国内外关于开放式创新的研究已经取得了许多成果,但在以下几个方面仍存在不足和研究空白:

\begin{itemize}
    \item \textbf{缺乏具体案例分析}:现有研究多集中于理论探讨和总体策略,缺乏对具体产品(如Apple Watch)开发过程中的开放式创新应用的详细分析。
    \item \textbf{应用效果评估不足}:关于开放式创新对产品成功的具体影响研究较少,尤其是定量评估方面。
    \item \textbf{跨行业对比研究不足}:不同企业和行业在开放式创新应用上的差异和共性研究较少,无法提供普遍适用的最佳实践。
\end{itemize}

本研究将通过详细分析Apple Watch的开发过程,探讨苹果公司在其中采用的开放式创新策略,评估其对产品成功的影响,
并与其他可穿戴设备进行对比分析,填补上述研究空白。
