% !TeX root = ../thuthesis-example.tex

\chapter{图表示例}

\section{插图}

图片通常在 \env{figure} 环境中使用 \cs{includegraphics} 插入,如图~\ref{fig:example} 的源代码。
建议矢量图片使用 PDF 格式,比如数据可视化的绘图;
照片应使用 JPG 格式;
其他的栅格图应使用无损的 PNG 格式。
注意,LaTeX 不支持 TIFF 格式;EPS 格式已经过时。

\begin{figure}
  \centering
  \includegraphics[width=0.5\linewidth]{example-image-a.pdf}
  \caption{Viral load and CD4+T cells changed over time (Zhang, San)}
  \figurenote{Viral load and CD4+T cells changed though time since HAART initiation in INR group (a) and IR group (b); the increase of naïve CD4+T cell count ($\increment n$) was significantly different between INR group and IR group (c); the increase of memory CD4+T cell count ($\increment n$) was similar in INR and IR groups (d). *, significant difference}
  \label{fig:example}
\end{figure}

如果一个图由两个或两个以上分图组成时,各分图分别以 (a)、(b)、(c)...... 作为图序,并须有分图题。
推荐使用 \pkg{subcaption} 宏包来处理, 比如图~\ref{fig:subfig-a} 和图~\ref{fig:subfig-b}。

\begin{figure}
  \centering
  \subcaptionbox{分图 A\label{fig:subfig-a}}
    {\includegraphics[width=0.35\linewidth]{example-image-a.pdf}}
  \subcaptionbox{分图 B\label{fig:subfig-b}}
    {\includegraphics[width=0.35\linewidth]{example-image-b.pdf}}
  \caption{多个分图的示例}
  \label{fig:multi-image}
\end{figure}



\section{表格}

表应具有自明性。为使表格简洁易读,尽可能采用三线表,如表~\ref{tab:three-line}。
三条线可以使用 \pkg{booktabs} 宏包提供的命令生成。

\begin{table}
  \centering
  \caption{Antibodies}
  \begin{tabular}{ccc}
    \toprule
    Antigen & Clone    & Manufacturer \\
    \midrule
    CD3     & 145-2C11 & eBioscience \\
    CD4     & GK1.5    & eBioscience \\
    CD8     & 53-6.7   & eBioscience \\
    \bottomrule
  \end{tabular}
  \tablenote{\raggedleft(Neefjes et al., 2011)}
  \label{tab:three-line}
\end{table}

表格如果有附注,尤其是需要在表格中进行标注时,可以使用 \pkg{threeparttable} 宏包。
研究生要求使用英文小写字母 a、b、c……顺序编号,本科生使用圈码 ①、②、③……编号。

\begin{table}
  \centering
  \begin{threeparttable}[c]
    \caption{Antibodies}
    \label{tab:three-part-table}
    \begin{tabular}{ccc}
      \toprule
      Antigen & Clone             & Manufacturer \\
      \midrule
      CD3     & 145-2C11\tnote{a} & eBioscience \\
      CD4     & GK1.5\tnote{b}    & eBioscience \\
      CD8     & 53-6.7            & eBioscience \\
      \bottomrule
    \end{tabular}
    \begin{tablenotes}
      \item Table note text.
    \end{tablenotes}
  \end{threeparttable}
\end{table}

如某个表需要转页接排,可以使用 \pkg{longtable} 宏包,需要在随后的各页上重复表的编号。
编号后跟表题(可省略)和“(续)”,置于表上方。续表均应重复表头。

\begin{longtable}{cccc}
    \caption{跨页长表格的表题}
    \label{tab:longtable} \\
    \toprule
    表头 1 & 表头 2 & 表头 3 & 表头 4 \\
    \midrule
  \endfirsthead
    \caption*{续表~\thetable\quad 跨页长表格的表题} \\
    \toprule
    表头 1 & 表头 2 & 表头 3 & 表头 4 \\
    \midrule
  \endhead
    \bottomrule
  \endfoot
  Row 1  & & & \\
  Row 2  & & & \\
  Row 3  & & & \\
  Row 4  & & & \\
  Row 5  & & & \\
  Row 6  & & & \\
  Row 7  & & & \\
  Row 8  & & & \\
  Row 9  & & & \\
  Row 10 & & & \\
\end{longtable}



\section{算法}

算法环境可以使用 \pkg{algorithms} 或者 \pkg{algorithm2e} 宏包。

\renewcommand{\algorithmicrequire}{\textbf{输入:}\unskip}
\renewcommand{\algorithmicensure}{\textbf{输出:}\unskip}

\begin{algorithm}
  \caption{Calculate $y = x^n$}
  \label{alg1}
  \small
  \begin{algorithmic}
    \REQUIRE $n \geq 0$
    \ENSURE $y = x^n$

    \STATE $y \leftarrow 1$
    \STATE $X \leftarrow x$
    \STATE $N \leftarrow n$

    \WHILE{$N \neq 0$}
      \IF{$N$ is even}
        \STATE $X \leftarrow X \times X$
        \STATE $N \leftarrow N / 2$
      \ELSE[$N$ is odd]
        \STATE $y \leftarrow y \times X$
        \STATE $N \leftarrow N - 1$
      \ENDIF
    \ENDWHILE
  \end{algorithmic}
\end{algorithm}
