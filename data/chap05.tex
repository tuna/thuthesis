% !TeX root = ../thuthesis-example.tex

\chapter{预期成果及可能的创新点}

\section{预期成果}

通过本研究,预期将取得以下成果:

\subsection{系统梳理Apple Watch开发过程中的开放式创新策略}

本研究将系统梳理Apple Watch在开发过程中的开放式创新策略,包括技术引进、合作开发、开放平台和用户参与等方面。
通过对这些策略的详细分析,将揭示苹果公司在产品开发中如何有效利用外部资源和合作伙伴,
提升创新能力和市场竞争力\citep{apple2024, davidson_assessing_2023,dahlander_how_2010}。

\subsection{评估开放式创新对Apple Watch成功的影响}

本研究将评估开放式创新对Apple Watch技术创新、市场表现和用户满意度的具体影响。
通过定性和定量分析,将揭示开放式创新在不同开发阶段的应用效果,评估其对产品成功的贡献程度\citep{gehani2016corporate,chesbrough_beyond_2006}。

\subsection{提出开放式创新在可穿戴设备开发中的最佳实践}

基于对Apple Watch开发过程中的开放式创新策略的研究,本研究将总结其成功经验,
提出开放式创新在可穿戴设备开发中的最佳实践和建议。这些建议不仅适用于苹果公司,
也对其他科技企业具有重要的参考价值\citep{chesbrough2016}。

\section{可能的创新点}

本研究可能的创新点包括:

\subsection{填补Apple Watch开发过程中的开放式创新应用研究的空白}

现有关于Apple Watch的研究主要集中在其市场表现和技术创新上,而对其开发过程中开放式创新的系统性研究相对较少。
本研究将填补这一研究空白,通过详细分析其开发过程中的开放式创新策略,提供系统性和实证性的研究成果\citep{apple2024,davidson_assessing_2023}。

\subsection{提供关于开放式创新在科技产品开发中的系统性分析}

本研究将通过对Apple Watch的案例分析,提供关于开放式创新在科技产品开发中的系统性分析和实证研究。
这将为其他企业在进行开放式创新时提供重要的理论和实践指导\citep{gehani2016corporate, dahlander_how_2010}。

\subsection{提出提升开放式创新效率的方法}

通过总结苹果公司在Apple Watch开发过程中应用开放式创新的经验和教训,本研究将提出提升开放式创新效率的方法。
这些方法将包括如何有效地选择合作伙伴、如何管理跨组织的合作以及如何利用用户反馈进行产品改进等\citep{chesbrough_beyond_2006, chesbrough2016}。
