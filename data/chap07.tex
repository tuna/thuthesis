\chapter{项目前景与未来工作}
\label{cha:future_work}

\section{项目前景}
目前项目团队已经实现上述所有功能,同时在数据获取方面,团队通过获得了几千万份法律文书,三千余部全国性法律,以及部分常用法学文本资料。这为我们的研究顺利进展做了很好的铺垫。

项目团队将目前前沿的自然语言处理技术及统计学习方法应用于法律领域,针对不同功能提出了不同的算法,实现了一个高效的、高准确率的法律案情分析平台。一方面,对于广大的无法学知识背景的非专业人才,实时的、证据充足的案情分析功能,可以让大家在遇到法律问题时,能够及时地有理有据地采取相应的法律措施,让自己避免陷入财产损失、权利被侵犯的困境中。另一方面,对于众多法律领域的从业人员来说,案情分析辅助平台可以帮助他们减少重复工作、提高工作效率,让自己的能力最大化。因此,该项目可以为许多人带来便利。因此项目在面向不同群体时,均可以起到很好的辅助作用,同时在大数据与人工智能高速发展的时代背景下,法律与信息技术、人工智能的结合也必然将成为时代趋势。


\section{未来工作}

目前,人工智能领域正在高速发展,而与此同时,法律智能的研究也吸引了计算机领域与法学领域学者的高度关注。团队尝试的工作也是受到许多限制的。因此,在未来,团队将会继续探索,用自然语言处理的技术去更好地解决相关的法律问题。
	我们将在以下几个方面进行更深层次的探索:
	
1.	自然语义处理技术在法律领域的可解释性应用。目前神经网络的一大缺点便是缺乏可解释性,在法律领域,预测结果的解释性是非常重要的,我们未来将在这样一个方向做进一步努力;

2.	拓宽任务种类,目前的项目中解决的任务中相对法学领域还相对基础,我们将加强与法学领域从业人员的沟通,进一步了解需求,拓展项目功能。

随着技术的不断发展,法学与人工智能的结合将渐渐成为趋势潮流,在越来越多的方面帮助到越来越多的人,我们将为此不断努力。

