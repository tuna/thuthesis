\section{研究背景}

在全球竞争日益激烈和技术迅速发展的背景下,企业必须不断推出新产品以维持和提升市场竞争力。新产品导入(New Product Introduction, NPI)过程是企业产品生命周期管理的关键环节,其成败直接影响新产品的市场表现和企业的盈利能力。

然而,现代NPI过程的复杂性日益增加,主要体现在以下几个方面:

\begin{enumerate}
  \item \textbf{多部门协同的复杂性}:NPI涉及研发、生产、供应链管理、市场营销等多个部门的协作。这些部门之间的沟通和协调对于项目成功至关重要,但也增加了管理的难度\cite{clark1991product}。

  \item \textbf{技术和市场的不确定性}:快速变化的技术环境和多变的市场需求使NPI过程充满不确定性,需要企业具备高效的响应和调整能力\cite{wheelwright1992revolutionizing}。

  \item \textbf{全球供应链的复杂性}:在全球化背景下,供应链涉及多个国家和地区,增加了协调难度和风险\cite{trent2003international}。

  \item \textbf{产品创新的复杂性}:消费者需求的多样化和个性化趋势,促使企业开发更加复杂和创新的产品\cite{tidd2013managing}。
\end{enumerate}

传统的项目管理方法,如瀑布式模型和线性流程,往往难以应对NPI过程中的高度复杂性和不确定性,可能导致项目延期、成本超支或质量问题\cite{cooper1994third}。

鉴于上述挑战,将\textbf{网络分析}方法应用于NPI过程成为一种新兴的研究方向。网络分析可以帮助理解复杂系统中各要素之间的关系和互动,为优化流程和提升协作效率提供科学依据\cite{newman2010networks}。

因此,探索基于网络分析的方法来优化NPI过程,具有重要的理论意义和实践价值。这不仅有助于企业更有效地管理新产品导入,还可提高其市场竞争力和创新能力。

\bibliographystyle{plain}
\bibliography{your_bib_file}
