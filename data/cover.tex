\thusetup{
  %******************************
  % 注意:
  %   1. 配置里面不要出现空行
  %   2. 不需要的配置信息可以删除
  %******************************
  %
  %=====
  % 秘级
  %=====
  secretlevel={秘密},
  secretyear={10},
  %
  %=========
  % 中文信息
  %=========
  ctitle={JudgeAI:法律智能案情分析平台},
  % cdegree={工学硕士},
  %cdepartment={计算机科学与技术系},
  %cmajor={计算机科学与技术},
  cauthor={肖朝军},
  %csupervisor={郑纬民教授},
  %cassosupervisor={陈文光教授}, % 副指导老师
  %ccosupervisor={某某某教授}, % 联合指导老师
  % 日期自动使用当前时间,若需指定按如下方式修改:
  % cdate={超新星纪元},
  %
  % 博士后专有部分
  % 这块比较复杂,需要分情况讨论:
  % 1. 学术型硕士
  %    edegree:必须为Master of Arts或Master of Science(注意大小写)
  %             “哲学、文学、历史学、法学、教育学、艺术学门类,公共管理学科
  %              填写Master of Arts,其它填写Master of Science”
  %    emajor:“获得一级学科授权的学科填写一级学科名称,其它填写二级学科名称”
  % 2. 专业型硕士
  %    edegree:“填写专业学位英文名称全称”
  %    emajor:“工程硕士填写工程领域,其它专业学位不填写此项”
  % 3. 学术型博士
  %    edegree:Doctor of Philosophy(注意大小写)
  %    emajor:“获得一级学科授权的学科填写一级学科名称,其它填写二级学科名称”
  % 4. 专业型博士
  %    edegree:“填写专业学位英文名称全称”
  %    emajor:不填写此项
  edegree={Doctor of Engineering},
  emajor={Computer Science and Technology},
  eauthor={Xue Ruini},
  esupervisor={Professor Zheng Weimin},
  eassosupervisor={Chen Wenguang},
  % 日期自动生成,若需指定按如下方式修改:
  % edate={December, 2005}
  %
  % 关键词用“英文逗号”分割
  % ckeywords={TeX, LaTeX, CJK, 模板, 论文},
  % ekeywords={TeX, LaTeX, CJK, template, thesis}
}

% 定义中英文摘要和关键字
\begin{cabstract}
  自十九大以来,国家不断强调深入推进依法治国、建设法制社会工作的开展。这也就要求每一个人都要有较强的法制观念,能够正确利用法律维护国家、自身的权益。但是,由于法律知识的缺乏、专业人员的稀缺,大部分人在遇到法律问题时,无法得到及时的帮助。据数据统计,2017年全国法院受理的案件数量高达2800万件,然而其中只有不足20\%的案件能够得到专业律师代理。因此,目前社会急缺法律专业人才来满足日益增长的法律服务需求。
  
	与此同时,如何在满足大量的法律需求的同时促进司法公正,也是建设法制社会进程中不可避免的重要问题。目前,国家已经通过司法公开化等诸多措施来保证每一个案件得到公平公正的审判,但是由于受到自身认知能力的局限性、法律条文的抽象性限制,不同的法官在判案时很难对案件做到毫无遗漏的了解和掌握。这很大程度上限制了司法公正的真正实现。
	
	本作品旨在解决上述问题。近几年深度学习技术的高速发展,使得人工智能得到越来越多学者的关注。很多问题也都得到了解决,其中序列标注、文本分类、机器阅读理解等任务得到了很大的突破。以此为基础,我们团队创新性地提出了法律案情分析平台——JudgeAI,该平台利用前沿自然语言处理技术来解决实际生活中的法律问题,实现了对案件的关键词标签抽取、案情事件抽取、判决预测、相关案件推荐、法律语义理解等功能。JudgeAI能够从全方面、多角度对案情描述进行分析,很好的满足了大众对于法律服务的需求,同时能够为专业人士提供统一的判案标准,推动司法公正。

\end{cabstract}

% 如果习惯关键字跟在摘要文字后面,可以用直接命令来设置,如下:
\ckeywords{法律;司法公正;人工智能;自然语言处理;案情分析}

\begin{eabstract}
	
\end{eabstract}


% \ekeywords{\TeX, \LaTeX, CJK, template, thesis}
