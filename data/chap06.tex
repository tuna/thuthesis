\chapter{研究计划}

\section{研究计划概述}
本研究预计将在12个月内完成,整个研究过程分为六个主要阶段:文献综述与理论构建、数据收集与案例研究、网络分析与模型构建、优化策略的提出与验证、论文撰写与修改,以及答辩准备与提交。每个阶段的时间安排和主要任务将在下文详细说明。

\section{研究时间安排}
\begin{table}[h!]
\centering
\begin{tabular}{|c|c|l|}
\hline
\textbf{时间段} & \textbf{阶段} & \textbf{主要任务} \\ \hline
第1-2个月 & 文献综述与理论构建 & \begin{tabular}[c]{@{}l@{}}- 系统收集与分析国内外相关文献\\ - 确立研究框架与假设\\ - 构建理论模型\end{tabular} \\ \hline
第3-4个月 & 数据收集与案例研究 & \begin{tabular}[c]{@{}l@{}}- 选择研究案例并进行企业合作\\ - 收集NPI过程中的任务依赖、资源分配等数据\\ - 进行初步数据分析\end{tabular} \\ \hline
第5-6个月 & 网络分析与模型构建 & \begin{tabular}[c]{@{}l@{}}- 使用网络分析工具(如Gephi)构建任务依赖网络\\ - 识别关键路径、瓶颈任务及风险点\\ - 开发资源分配与优化模型\end{tabular} \\ \hline
第7-8个月 & 优化策略的提出与验证 & \begin{tabular}[c]{@{}l@{}}- 提出基于网络分析的NPI流程优化策略\\ - 通过模拟实验或实地应用验证优化方案\\ - 收集反馈并调整优化策略\end{tabular} \\ \hline
第9-10个月 & 论文撰写与修改 & \begin{tabular}[c]{@{}l@{}}- 撰写研究论文的各个部分\\ - 对论文进行反复修改与完善\\ - 完成论文的初稿与最终定稿\end{tabular} \\ \hline
第11-12个月 & 答辩准备与提交 & \begin{tabular}[c]{@{}l@{}}- 准备答辩材料(如PPT)\\ - 进行答辩预演\\ - 完成论文提交与答辩\end{tabular} \\ \hline
\end{tabular}
\caption{研究时间安排表}
\label{tab:time_plan}
\end{table}

\section{各阶段主要任务详解}

\subsection{文献综述与理论构建(第1-2个月)}
本阶段的主要任务是系统收集和分析国内外相关文献,明确网络分析在NPI流程管理中的应用现状与挑战,并基于复杂性管理理论,构建本研究的理论框架。这一阶段还将提出研究假设和问题,为后续研究奠定基础。

\subsection{数据收集与案例研究(第3-4个月)}
在本阶段,将选择一个或多个实际NPI项目作为研究案例,通过企业内部合作和实地调研,收集项目中的任务依赖关系、资源分配情况和项目进度数据。这些数据将用于后续的网络建模和分析。

\subsection{网络分析与模型构建(第5-6个月)}
利用前期收集到的数据,本阶段将通过网络分析工具构建任务依赖网络,识别项目中的关键路径、瓶颈任务及潜在风险点。此外,还将开发资源分配与优化模型,为企业提供科学的管理工具。

\subsection{优化策略的提出与验证(第7-8个月)}
基于网络分析的结果,本阶段将提出NPI流程优化策略,并通过模拟实验或实地应用对这些策略进行验证。反馈结果将用于进一步调整和完善优化方案,确保其在不同项目环境中的适用性。

\subsection{论文撰写与修改(第9-10个月)}
本阶段将撰写研究论文的各个部分,包括引言、文献综述、研究方法、研究结果和讨论等章节。论文初稿完成后,将进行多次修改和完善,直至形成最终定稿。

\subsection{答辩准备与提交(第11-12个月)}
在研究的最后阶段,将集中准备答辩材料(如PPT),并进行答辩预演,确保能够清晰、有效地展示研究成果。同时,将按照学校要求完成论文的提交工作,并顺利通过答辩。

\section{本章小结}
本章详细介绍了本研究的时间安排和各个阶段的主要任务。通过合理的时间规划和步骤安排,本研究将系统地探讨网络分析在NPI流程管理中的应用,并提出切实可行的优化策略,最终形成高质量的学术论文。

