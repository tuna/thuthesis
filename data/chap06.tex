\chapter{预期成果}

本研究旨在探索基于复杂系统理论和网络分析方法的新产品导入(NPI)过程优化策略,预期将取得以下成果:

\section{理论贡献}

\begin{enumerate}
  \item \textbf{构建 NPI 过程的复杂系统网络模型}:

  提出了将 NPI 过程视为复杂系统的方法,建立了涵盖任务、组织和信息流等多层次要素的网络模型,丰富了复杂系统理论在工程管理领域的应用。

  \item \textbf{发展 NPI 过程的网络分析方法}:

  将复杂网络分析的方法应用于 NPI 过程,提出了适用于 NPI 网络特征分析的指标和算法,拓展了网络科学在工程管理中的应用范围。

  \item \textbf{揭示 NPI 过程中的关键因素和机制}:

  通过网络分析,揭示了影响 NPI 成功的关键节点、关键路径和协同机制,为理解 NPI 过程的复杂性提供了新的视角。

\end{enumerate}

\section{实践意义}

\begin{enumerate}
  \item \textbf{提供 NPI 过程优化的实用方法}:

  基于网络分析的结果,提出了具体的 NPI 过程优化策略,为企业提升新产品导入效率和成功率提供了可操作的工具和方法。

  \item \textbf{提高企业协同管理能力}:

  通过优化 NPI 网络结构,改善部门间的协作和信息共享,增强企业应对市场变化和技术挑战的能力。

  \item \textbf{支持决策的可视化和量化分析}:

  利用网络模型和分析结果,为企业管理者提供直观的决策支持工具,帮助识别潜在问题和改进机会。

\end{enumerate}

\section{应用价值}

\begin{enumerate}
  \item \textbf{案例研究的实践应用}:

  通过对实际企业 NPI 过程的分析和优化,为企业实践提供了可借鉴的经验和方法,有助于提升企业的 NPI 管理水平。

  \item \textbf{开发相关工具或指南}:

  基于本研究的方法和模型,编制 NPI 过程网络分析和优化的指导手册或工具,帮助企业实施和应用本研究成果。

  \item \textbf{促进学科知识的综合应用}:

  本研究融合了复杂系统科学、网络科学和工程管理学科的知识,促进了学科间的交叉融合,为工程管理实践提供了新的思路。

\end{enumerate}