
\chapter{研究背景}

在全球竞争日益激烈和技术迅速发展的背景下,企业必须不断推出新产品以维持和提升市场竞争力。
新产品导入(New Product Introduction, NPI)过程是企业产品生命周期管理的关键环节,其成败直接影响新产品的市场表现和企业的盈利能力。

\section{复杂系统的定义与特征}

复杂系统是由大量相互作用的组成部分构成,其整体行为难以通过简单地叠加各部分的行为来预测\citep{anderson1972more}。
复杂系统的主要特征包括\citep{mitchell2009complexity}:

\begin{itemize}
  \item \textbf{非线性}:系统的输出与输入不成比例,微小的变化可能导致巨大影响。
  \item \textbf{自组织性}:系统能够自发地形成有序结构或行为模式,无需外部指令。
  \item \textbf{涌现性}:整体系统表现出单个部分所不具备的性质或功能。
  \item \textbf{适应性}:系统能够根据环境变化进行调整和进化。
\end{itemize}

复杂系统理论为理解和管理高度复杂、动态变化的系统提供了理论基础\citep{holland2006studying}。

\section{NPI 过程的复杂性}

NPI 过程具备典型的复杂系统特征,主要体现在以下方面:

\begin{enumerate}
  \item \textbf{多元参与者的相互作用}:涉及研发、生产、供应链、市场营销等多个部门和外部合作伙伴,各参与者之间存在高度的相互依赖和互动\citep{kusumanoKimCLARKTakahiro1992}。
  
  \item \textbf{非线性流程}:NPI 过程中的决策和反馈机制使其呈现非线性特征,局部的变化可引发全局性的影响\citep{wheelwright1992revolutionizing}。
  
  \item \textbf{不确定性和动态性}:技术创新和市场需求的快速变化增加了 NPI 过程的不确定性,需要持续的调整和适应\citep{trent2003international}。
  
  \item \textbf{涌现特性}:通过各部门的协同工作,NPI 过程可能产生新的知识和创新,这些是单个部门无法独立实现的\citep{tidd2013managing}。
\end{enumerate}

由于上述复杂性,传统的线性和阶段式项目管理方法(如瀑布模型)难以有效管理 NPI 过程\citep{cooper1994third}。
因此,引入复杂系统理论和\textbf{网络分析}方法,可以更好地理解 NPI 过程中的复杂互动关系,为优化流程和提高协作效率提供新的思路\citep{newman2010networks}。

\section{复杂系统与网络分析的应用}

网络分析是研究复杂系统的重要工具,通过构建节点和边的关系,揭示系统内在的结构和动态特性\citep{barabasiNetworkScience2013}。在 NPI 过程中,网络分析可以用于:

\begin{itemize}
  \item 分析部门之间的协作关系,识别关键节点和瓶颈。
  \item 评估信息流动和资源分配的效率。
  \item 模拟不同管理策略对整体系统的影响。
\end{itemize}

因此,探索基于复杂系统理论和网络分析的方法来优化 NPI 过程,具有重要的理论意义和实践价值。这有助于企业在高度竞争和不确定的环境中,提高新产品导入的成功率和效率。