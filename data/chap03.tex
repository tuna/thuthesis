\chapter{项目创新点与意义}
\label{cha: significance}

\section{项目创新点}

在学术研究上,团队尝试应用自然语言处理技术解决实际生活中的法学问题。通过长时间的探索与交流,我们总结并提出了多个常见的法学任务,并将其形式化为能够利用自然语言处理模型进行处理的常见任务。并且在多个任务上,团队提出了创新性的模型,达到了不错的效果。

另一方面,通过与法律从业人员长时间的沟通交流,团队提出的任务都是建立在切实解决实际问题的基础之上的。我们致力于打造一个功能全面、技术可靠的案情分析平台。相比于市面上现有的一些法律产品,我们平台的功能更加全面、实用,结果更加可靠。

总结而言,项目有以下亮点与创新:

1)	尝试利用自然语言处理的最前沿的技术解决相关的法律问题。通过对案由预测、相关法条预测、刑期预测、关键词抽取、类案推荐等功能的实现,平台可以帮助法律领域从业人员减免重复工作,成为辅助其工作的好工具;同时也可以为大家身边遇到的法律问题提供的解决方案,成为无法学背景的非专业人士的好帮手。
2)	提出了全面的法学基础任务,能够对每一段案情进行多角度全方位的分析,更好的满足了用户需求。通过与从业人员的多次的深入交流,我们获知了他们在工作中碰到最多的几大问题,并进行针对性的解决,真正做到了充分了解用户需求。
3)	在判决预测模块,我们提出了一个能够捕捉子任务间依赖关系的多任务学习模型,超过了以往模型,实现了state-of-the-art效果。在我们提出的几个任务之间,往往有着很强的依赖关系,例如案由与法条之间具有很强的映射关系。模型通过捕捉这些子任务之间的映射关系,提升其效果。
4)	首次在关键词抽取任务上运用并改进Lattice-LSTM模型。克服了词级别模型过度依赖于中文分词效果、字级别模型语义信息不足的缺点,在传统的序列标注模型上实现了大幅提升。
5)	在类案检索模块,我们提出了一个基于关键词抽取、案件语义理解的模型,做到了在语义层面上的相似性检索。模型首先抽取案情关键词标签,通过标签缩小候选的相似文本集合,再进一步通过不同文章的文章向量之间的距离来衡量案情的相似程度。相比于传统搜索引擎的文本相似性检索,此搜索模块可以做到真正的语义相似性检索。

\section{项目意义}




