\documentclass[type=master]{thuthesis}
%\documentclass[type=bachelor, fontset=windowsold, AutoFakeBold=2.5]{thuthesis} % 本科按此设置
% 选项:
%   type=[bachelor|master|doctor|postdoctor], % 必选
%   secret,                                   % 可选
%   pifootnote,                               % 可选(建议打开)
%   openany|openright,                        % 可选,基本不用
%   arial,                                    % 可选,基本不用
%   arialtoc,                                 % 可选,基本不用
%   arialtitle                                % 可选,基本不用

% 所有其它可能用到的包都统一放到这里了,可以根据自己的实际添加或者删除。
\usepackage{thuthesis}

% 定义所有的图片文件在 figures 子目录下
\graphicspath{{figures/}}

% 可以在这里修改配置文件中的定义。导言区可以使用中文。
% \def\myname{薛瑞尼}

\begin{document}

%%% 封面部分
\frontmatter
\thusetup{
  %******************************
  % 注意:
  %   1. 配置里面不要出现空行
  %   2. 不需要的配置信息可以删除
  %******************************
  %
  %=====
  % 秘级
  %=====
  secretlevel={秘密},
  secretyear={10},
  %
  %=========
  % 中文信息
  %=========
  ctitle={JudgeAI:法律智能案情分析平台},
  % cdegree={工学硕士},
  %cdepartment={计算机科学与技术系},
  %cmajor={计算机科学与技术},
  cauthor={肖朝军},
  %csupervisor={郑纬民教授},
  %cassosupervisor={陈文光教授}, % 副指导老师
  %ccosupervisor={某某某教授}, % 联合指导老师
  % 日期自动使用当前时间,若需指定按如下方式修改:
  % cdate={超新星纪元},
  %
  % 博士后专有部分
  % 这块比较复杂,需要分情况讨论:
  % 1. 学术型硕士
  %    edegree:必须为Master of Arts或Master of Science(注意大小写)
  %             “哲学、文学、历史学、法学、教育学、艺术学门类,公共管理学科
  %              填写Master of Arts,其它填写Master of Science”
  %    emajor:“获得一级学科授权的学科填写一级学科名称,其它填写二级学科名称”
  % 2. 专业型硕士
  %    edegree:“填写专业学位英文名称全称”
  %    emajor:“工程硕士填写工程领域,其它专业学位不填写此项”
  % 3. 学术型博士
  %    edegree:Doctor of Philosophy(注意大小写)
  %    emajor:“获得一级学科授权的学科填写一级学科名称,其它填写二级学科名称”
  % 4. 专业型博士
  %    edegree:“填写专业学位英文名称全称”
  %    emajor:不填写此项
  edegree={Doctor of Engineering},
  emajor={Computer Science and Technology},
  eauthor={Xue Ruini},
  esupervisor={Professor Zheng Weimin},
  eassosupervisor={Chen Wenguang},
  % 日期自动生成,若需指定按如下方式修改:
  % edate={December, 2005}
  %
  % 关键词用“英文逗号”分割
  % ckeywords={TeX, LaTeX, CJK, 模板, 论文},
  % ekeywords={TeX, LaTeX, CJK, template, thesis}
}

% 定义中英文摘要和关键字
\begin{cabstract}
  自十九大以来,国家不断强调深入推进依法治国、建设法制社会工作的开展。这也就要求每一个人都要有较强的法制观念,能够正确利用法律维护国家、自身的权益。但是,由于法律知识的缺乏、专业人员的稀缺,大部分人在遇到法律问题时,无法得到及时的帮助。据数据统计,2017年全国法院受理的案件数量高达2800万件,然而其中只有不足20\%的案件能够得到专业律师代理。因此,目前社会急缺法律专业人才来满足日益增长的法律服务需求。
  
	与此同时,如何在满足大量的法律需求的同时促进司法公正,也是建设法制社会进程中不可避免的重要问题。目前,国家已经通过司法公开化等诸多措施来保证每一个案件得到公平公正的审判,但是由于受到自身认知能力的局限性、法律条文的抽象性限制,不同的法官在判案时很难对案件做到毫无遗漏的了解和掌握。这很大程度上限制了司法公正的真正实现。
	
	本作品旨在解决上述问题。近几年深度学习技术的高速发展,使得人工智能得到越来越多学者的关注。很多问题也都得到了解决,其中序列标注、文本分类、机器阅读理解等任务得到了很大的突破。以此为基础,我们团队创新性地提出了法律案情分析平台——JudgeAI,该平台利用前沿自然语言处理技术来解决实际生活中的法律问题,实现了对案件的关键词标签抽取、案情事件抽取、判决预测、相关案件推荐、法律语义理解等功能。JudgeAI能够从全方面、多角度对案情描述进行分析,很好的满足了大众对于法律服务的需求,同时能够为专业人士提供统一的判案标准,推动司法公正。

\end{cabstract}

% 如果习惯关键字跟在摘要文字后面,可以用直接命令来设置,如下:
\ckeywords{法律;司法公正;人工智能;自然语言处理;案情分析}

\begin{eabstract}
	
\end{eabstract}


% \ekeywords{\TeX, \LaTeX, CJK, template, thesis}

% 如果使用授权说明扫描页,将可选参数中指定为扫描得到的 PDF 文件名,例如:
% \makecover[scan-auth.pdf]
\makecover

%% 目录
\tableofcontents

%% 符号对照表
\input{data/denotation}


%%% 正文部分
\mainmatter
\chapter{项目背景}
% \label{cha:intro}
\label{cha:background}


\section{社会背景}
法律是治国之重器,法律表明了国家权力的运行和国家意志的体现。法律在规范个人、保障社会秩序方面起到了非常重要的作用,同时法律也是我们每一个人维护自身权益的保证。因此,法律对于国家、社会、个人都是相当重要的。自新中国成立以来,全国上下各个司法机构、学术领域产生了大量的法律文书、法学期刊等文件。这些法学文件一直以来被法学领域的学者、法官、律师重视,成为了他们工作中非常重要的一个部分。自国家全面推行依法治国以来,法律在我们日常生活中扮演着越来越重要的角色。但是,当人们在生活中遇到法律问题时,却常常因为缺乏专业知识而不知道如何进行维权。同时,法学领域的学者、工作人员在面对浩如烟海的法律文本时,也常常因为缺乏好的检索工具和阅读系统,而需要花费大量的时间来查找相关材料,大大降低了工作效率。

据统计,近几年来,中国每天发生的法律案件日益增加,2018年人民法院受理的案件高达2800万件,创下历史新高 ;全国共有12万法官,平均每个法官一年审理案件233件;全国共拥有36.5万的职业律师,一年代理的诉讼案件达到465万件,却不足一年案件总数的20\%。上述数据显示,法律服务在中国有着庞大的市场规模。但是由于法律人才的供应不足,市场上对法律服务的需求始终没有办法得到满足。 

在另一方面,技术的创新给我们的生活带来了许许多多的改变,其中表现最突出的便是人工智能与大数据的技术。伴随着深度学习的发展,人工智能中的许多领域在近几年来取得了很大的进步。自然语言处理技术便是其中的代表,在该领域中文本分类、知识图谱、信息抽取、机器翻译等任务上,现有模型纷纷打破了前人的种种记录。技术的创新带来的是生活的改变,而这样的改变也逐渐渗透到法学领域。2018年,司法部提出“加快‘数字法治、智慧司法’建设”的口号 ,这也让法学学者也越发注重人工智能在司法领域的应用。


\section{商业背景}
随着法律智能受到社会的广泛关注,各大有关公司也都推出了相应的产品。其中最具代表性的法律工具网站有:中国裁判文书网、北大法宝、法信网、法咚咚、搜狗法律等。这些产品大多以基于词匹配的案件文书检索为核心。受限于统计方法的检索能力,这些产品大多面向法律专业人士,旨在为他们提供基础的法律服务。


在这样的契机之下,团队构建了一个案情分析平台——JudgeAI,尝试将自然语言技术运用到法学领域的各个问题之上,旨在利用自然语言处理的技术的方法来解决实际法律问题,为大家的工作、生活提供便利。我们实现了对案情的全方面多角度的分析,在输入一段文本形式案情描述之后,我们平台将提供:案件的案由/罪名预测、判案要素预测、相关法条预测、相似案件检索、关键词提取和相关问题推荐的功能。以此让用户能够尽快捕捉到文本中的有用信息,并作出相应的判断,从而达到辅助判决的作用。相比于已有的产品、算法,我们团队在这些任务上提升了对长文本语义的理解,超过以往模型,达到了更高的预测精度。



% !TeX root = ../thuthesis-example.tex

\chapter{图表示例}

\section{插图}

图片通常在 \env{figure} 环境中使用 \cs{includegraphics} 插入,如图~\ref{fig:example} 的源代码。
建议矢量图片使用 PDF 格式,比如数据可视化的绘图;
照片应使用 JPG 格式;
其他的栅格图应使用无损的 PNG 格式。
注意,LaTeX 不支持 TIFF 格式;EPS 格式已经过时。

\begin{figure}
  \centering
  \includegraphics[width=0.5\linewidth]{example-image-a.pdf}
  \caption*{国外的期刊习惯将图表的标题和说明文字写成一段,需要改写为标题只含图表的名称,其他说明文字以注释方式写在图表下方,或者写在正文中。}
  \caption{示例图片标题}
  \label{fig:example}
\end{figure}

若图或表中有附注,采用英文小写字母顺序编号,附注写在图或表的下方。
国外的期刊习惯将图表的标题和说明文字写成一段,需要改写为标题只含图表的名称,其他说明文字以注释方式写在图表下方,或者写在正文中。

如果一个图由两个或两个以上分图组成时,各分图分别以 (a)、(b)、(c)...... 作为图序,并须有分图题。
推荐使用 \pkg{subcaption} 宏包来处理, 比如图~\ref{fig:subfig-a} 和图~\ref{fig:subfig-b}。

\begin{figure}
  \centering
  \subcaptionbox{分图 A\label{fig:subfig-a}}
    {\includegraphics[width=0.35\linewidth]{example-image-a.pdf}}
  \subcaptionbox{分图 B\label{fig:subfig-b}}
    {\includegraphics[width=0.35\linewidth]{example-image-b.pdf}}
  \caption{多个分图的示例}
  \label{fig:multi-image}
\end{figure}



\section{表格}

表应具有自明性。为使表格简洁易读,尽可能采用三线表,如表~\ref{tab:three-line}。
三条线可以使用 \pkg{booktabs} 宏包提供的命令生成。

\begin{table}
  \centering
  \caption{三线表示例}
  \begin{tabular}{ll}
    \toprule
    文件名          & 描述                         \\
    \midrule
    thuthesis.dtx   & 模板的源文件,包括文档和注释 \\
    thuthesis.cls   & 模板文件                     \\
    thuthesis-*.bst & BibTeX 参考文献表样式文件    \\
    \bottomrule
  \end{tabular}
  \label{tab:three-line}
\end{table}

表格如果有附注,尤其是需要在表格中进行标注时,可以使用 \pkg{threeparttable} 宏包。
研究生要求使用英文小写字母 a、b、c……顺序编号,本科生使用圈码 ①、②、③……编号。

\begin{table}
  \centering
  \begin{threeparttable}[c]
    \caption{带附注的表格示例}
    \label{tab:three-part-table}
    \begin{tabular}{ll}
      \toprule
      文件名                 & 描述                         \\
      \midrule
      thuthesis.dtx\tnote{a} & 模板的源文件,包括文档和注释 \\
      thuthesis.cls\tnote{b} & 模板文件                     \\
      thuthesis-*.bst        & BibTeX 参考文献表样式文件    \\
      \bottomrule
    \end{tabular}
    \begin{tablenotes}
      \item [a] 可以通过 xelatex 编译生成模板的使用说明文档;
        使用 xetex 编译 \file{thuthesis.ins} 时则会从 \file{.dtx} 中去除掉文档和注释,得到精简的 \file{.cls} 文件。
      \item [b] 更新模板时,一定要记得编译生成 \file{.cls} 文件,否则编译论文时载入的依然是旧版的模板。
    \end{tablenotes}
  \end{threeparttable}
\end{table}

如某个表需要转页接排,可以使用 \pkg{longtable} 宏包,需要在随后的各页上重复表的编号。编号后跟表题(可省略)和“(续)”,置于表上方。续表均应重复表头。

\begin{longtable}{cccc}
    \caption{跨页长表格的表题}
    \label{tab:longtable} \\
    \toprule
    表头 1 & 表头 2 & 表头 3 & 表头 4 \\
    \midrule
  \endfirsthead
    \caption*{续表~\thetable\quad 跨页长表格的表题} \\
    \toprule
    表头 1 & 表头 2 & 表头 3 & 表头 4 \\
    \midrule
  \endhead
    \bottomrule
  \endfoot
  Row 1  & & & \\
  Row 2  & & & \\
  Row 3  & & & \\
  Row 4  & & & \\
  Row 5  & & & \\
  Row 6  & & & \\
  Row 7  & & & \\
  Row 8  & & & \\
  Row 9  & & & \\
  Row 10 & & & \\
\end{longtable}

如果表格的内容长度超过了页面长度,可以使用 \pkg{tabularx} 宏包实现表格内容的自动换行,自动换行的列对齐方式使用X表示。

\begin{table}
  \centering
  \caption{表格内容自动换行示例}
  \begin{tabularx}{\textwidth}{lX}
    \toprule
    文件名          & 描述                         \\
    \midrule
    thuthesis.dtx   & 模板的源文件,包括文档和注释。模板的源文件,包括文档和注释。模板的源文件,包括文档和注释。模板的源文件,包括文档和注释。模板的源文件,包括文档和注释。模板的源文件,包括文档和注释。 \\
    thuthesis.cls   & 模板文件                     \\
    thuthesis-*.bst & BibTeX 参考文献表样式文件    \\
    \bottomrule
  \end{tabularx}
  \label{tab:tab-auto-wrap}
\end{table}

\section{算法}

算法环境可以使用 \pkg{algorithms} 或者 \pkg{algorithm2e} 宏包。

\renewcommand{\algorithmicrequire}{\textbf{输入:}\unskip}
\renewcommand{\algorithmicensure}{\textbf{输出:}\unskip}

\begin{algorithm}
  \caption{Calculate $y = x^n$}
  \label{alg1}
  \small
  \begin{algorithmic}
    \REQUIRE $n \geq 0$
    \ENSURE $y = x^n$

    \STATE $y \leftarrow 1$
    \STATE $X \leftarrow x$
    \STATE $N \leftarrow n$

    \WHILE{$N \neq 0$}
      \IF{$N$ is even}
        \STATE $X \leftarrow X \times X$
        \STATE $N \leftarrow N / 2$
      \ELSE[$N$ is odd]
        \STATE $y \leftarrow y \times X$
        \STATE $N \leftarrow N - 1$
      \ENDIF
    \ENDWHILE
  \end{algorithmic}
\end{algorithm}



%%% 其它部分
\backmatter

%% 本科生要这几个索引,研究生不要。选择性留下。
% 插图索引
\listoffigures
% 表格索引
\listoftables
% 公式索引
\listofequations


%% 参考文献
% 注意:至少需要引用一篇参考文献,否则下面两行可能引起编译错误。
% 如果不需要参考文献,请将下面两行删除或注释掉。
\bibliographystyle{thuthesis-numerical}
\bibliography{ref/refs}


%% 致谢
%%% Local Variables:
%%% mode: latex
%%% TeX-master: "../main"
%%% End:

\begin{ack}
  衷心感谢导师 xxx 教授和物理系 xxx 副教授对本人的精心指导。他们的言传身教将使
  我终生受益。

  在美国麻省理工学院化学系进行九个月的合作研究期间,承蒙 xxx 教授热心指导与帮助,不
  胜感激。感谢 xx 实验室主任 xx 教授,以及实验室全体老师和同学们的热情帮助和支
  持!本课题承蒙国家自然科学基金资助,特此致谢。

  感谢 \thuthesis,它的存在让我的论文写作轻松自在了许多,让我的论文格式规整漂亮了
  许多。
\end{ack}


%% 附录
\begin{appendix}
\input{data/appendix01}
\end{appendix}

%% 个人简历
% !TeX root = ../thuthesis-example.tex

\begin{resume}

  \section*{个人简历}

  197× 年 ×× 月 ×× 日出生于四川××县。

  1992 年 9 月考入××大学化学系××化学专业,1996 年 7 月本科毕业并获得理学学士学位。

  1996 年 9 月免试进入清华大学化学系攻读××化学博士至今。


  \section*{在学期间完成的相关学术成果}

  \subsection{学术论文}

  \begin{achievements}
    \item Yang Y, Ren T L, Zhang L T, et al. Miniature microphone with silicon-based ferroelectric thin films[J]. Integrated Ferroelectrics, 2003, 52:229-235.
    \item 杨轶, 张宁欣, 任天令, 等. 硅基铁电微声学器件中薄膜残余应力的研究[J]. 中国机械工程, 2005, 16(14):1289-1291.
    \item 杨轶, 张宁欣, 任天令, 等. 集成铁电器件中的关键工艺研究[J]. 仪器仪表学报, 2003, 24(S4):192-193.
    \item Yang Y, Ren T L, Zhu Y P, et al. PMUTs for handwriting recognition. In press[J]. (已被Integrated Ferroelectrics录用)
  \end{achievements}


  \subsection{专利}

  \begin{achievements}
    \item 任天令, 杨轶, 朱一平, 等. 硅基铁电微声学传感器畴极化区域控制和电极连接的方法: 中国, CN1602118A[P]. 2005-03-30.
    \item Ren T L, Yang Y, Zhu Y P, et al. Piezoelectric micro acoustic sensor based on ferroelectric materials: USA, No.11/215, 102[P]. (美国发明专利申请号.)
  \end{achievements}

  \subsection*{3  奖项}

  \begin{achievements}
    \item 任天令, 杨轶, 朱一平, 等. 硅基铁电微声学传感器畴极化区域控制和电极连接的方法: 中国, CN1602118A[P]. 2005-03-30.
    \item Ren T L, Yang Y, Zhu Y P, et al. Piezoelectric micro acoustic sensor based on ferroelectric materials: USA, No.11/215, 102[P]. (美国发明专利申请号.)
  \end{achievements}

\end{resume}


%% 本科生进行格式审查是需要下面这个表格,答辩可能不需要。选择性留下。
% 综合论文训练记录表
\includepdf[pages=-]{scan-record.pdf}
\end{document}
