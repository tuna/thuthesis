%D \module
%D   [       file=t-thuthesis,
%D        version=0.1,
%D          title=Tsinghua University \CONTEXT\ Template,
%D       subtitle=Dissertation,
%D         author=Ruini Xue,
%D           date=\currentdate,
%D      copyright=Public Domain]

\writestatus{loading}{Tsinghua University \CONTEXT\ Template}

\unprotect

\startmodule[thuthesis]
\def\thuthesis{{\sc ThuThesis}}

% use module 
% \usemodule [bib]

% setup layout
\setuppapersize[A4][A4]
\setuplayout[width=fit,
%	     height=middle,
             leftmargin=3.2cm,
	     rightmargin=3.2cm,
             backspace=4cm,
             topspace=2cm,
             headerdistance=.5cm,
             footerdistance=.5cm,
	     leftmargindistance=.5cm,
             rightmargindistance=.5cm,
             header=20pt,
             footer=20pt]
\setuppagenumbering[alternative=doublesided,
                    style=\tfx,
                    location={footer,middle}]

% setup font
\usetypescriptfile [zhfonts]
\usetypescript [thuthesis]
\setupbodyfont [thuthesis, 12pt]
\setupbodyfontenvironment [default] [em=italic]

\def\thudefinefontsize{%
  \dodoubleargument\dothudefinefontsize}

% #1: font name 
% #2: default size
\def\dothudefinefontsize[#1][#2]{%
  \@EA\unexpanded\@EA\def\csname #1\endcsname{%
    \@EA\dosingleempty\csname do#1\endcsname}
   % ##1: line space
  \@EA\unexpanded\@EA\def\csname do#1\endcsname[##1]{%
    \iffirstargument % NOTE: this must be tested first!
      \switchtobodyfont[#2]
      \setupinterlinespace[line=##1]
    \else
      \switchtobodyfont[#2]
    \fi}}
	
\thudefinefontsize [chuhao]    [42bp]
\thudefinefontsize [xiaochu]   [36bp]
\thudefinefontsize [yihao]     [26bp]
\thudefinefontsize [xiaoyi]    [24bp]
\thudefinefontsize [erhao]     [22bp]
\thudefinefontsize [xiaoer]    [18bp]
\thudefinefontsize [sanhao]    [16bp]
\thudefinefontsize [xiaosan]   [15bp]
\thudefinefontsize [sihao]     [14bp]
\thudefinefontsize [banxiaosi] [13bp]
\thudefinefontsize [xiaosi]    [12bp]
\thudefinefontsize [dawu]      [11bp]
\thudefinefontsize [wuhao]     [10.5bp]
\thudefinefontsize [xiaowu]    [9bp]
\thudefinefontsize [liuhao]    [7.5bp]
\thudefinefontsize [xiaoliu]   [6.5bp]
\thudefinefontsize [qihao]     [5.5bp]
\thudefinefontsize [bahao]     [5bp]

% setup titles
\setupheads
  [indentnext=yes] 


\definepagebreak
    [mychapterpagebreak]
    [yes,header,right] % [pagebreak=yes,header,footer,openwhere=right]
\setuphead
   [title,chapter]
   [align=center,
    before={\blank[force,20bp]},
    after={\blank[24bp]},
    style={\sanhao\bf},
    page=mychapterpagebreak]
 %  [page=Mychapterpagebreak,header=empty,footer=empty]

\setuphead
  [section,subject]
  [before={\blank[24bp]},
   after={\blank[6bp]},
   style={\sihao\bf}]

\setuphead
  [subsection,subsubject]
  [before={\blank[16bp]},
   after={\blank[6bp]},
   style={\xiaosi\bf}]

\setuphead
  [subsubsection,subsubsubject]
  [before={\blank[12bp]},
   after={\blank[6bp]},
   style={\xiaosi\bf}]

\setupheader % todo: what's this?
  [text]
  [frame=off, bottomframe=on, style=\tfx]

\startsectionblockenvironment[frontpart]
  \setuppagenumbering[conversion=Romannumerals]
  \setuppagenumber[number=1]
  \setups [FrontHeaderFooter]
\stopsectionblockenvironment

\startsectionblockenvironment[bodypart]
  \setuppagenumbering[conversion=number]
  \setuppagenumber[number=1]
  \setups [BodyHeaderFooter]
\stopsectionblockenvironment

\startsectionblockenvironment[backpart]
  % todo
\stopsectionblockenvironment

\startsectionblockenvironment[appendix]
  % todo:
\stopsectionblockenvironment

\iffalse
% hack to remove excess blank pages in the end. needed because front- and back-matter usage
% makes the system assume it is a book and therefore it adds pages to make the number even
\setupsectionblock[frontpart][page=no] %todo: number=yes|no?
\setupsectionblock[bodypart][page=no]
\setupsectionblock[appendix][page=no]
\setupsectionblock[backpart][page=no]
\fi

% setup chinese
\setuplabeltext [chapter={第\;,\;章}]
% \setuplabeltext [en] [section={第\;,\;节}]
\setuplabeltext [figure=图\;]
\setuplabeltext [table=表\;]
% \setupheadtext [en] [pubs={\bfc 参考文献}]
\setupheadtext [content=目录]

% setup figure directory
% 图片目录
\setupexternalfigures[directory={./figures}]

%% 目录 %%
\def\ChapterNumber#1{\doiftext{#1}{第\;#1\;章\quad}}
\setuplist
    [chapter]
    [alternative=c,
     before={\page[preference]\blank},
     after=\blank,
     style=\bf,
     width=fit,
     pagestyle=bold,
     pagenumber=yes,
     numbercommand=\ChapterNumber]

\setuplist
    [section]
    [alternative=c,
     style=normal,
     pagestyle=\itx]


% 页眉页脚
\setupcombinedlist[content][level=3,alternative=c]
\startsetups FrontHeaderFooter
  \setupheadertexts 
    [{\inframed[width=\textwidth,frame=off,bottomframe=on]{\getmarking[title]}}]
  \setupfootertexts [pagenumber]
\stopsetups

\startsetups BodyHeaderFooter
  \def\CurrentChapter{%
    第 \headnumber[chapter]\ 章\hbox to 1em{} \getmarking[chapter]}
  \def\CurrentSection{%
    \headnumber[section] \getmarking[section]}
%  \setupheadertexts[text]
%                   [\CurrentChapter][]
%                   [][\CurrentSection]
  \setupheadertexts [{\inframed[width=\textwidth,frame=off,bottomframe=on]{\CurrentChapter}}]
  \setupfootertexts [pagenumber]
  \setupheader [style=\bfx]
  \setupfooter [style=\bfx]
\stopsetups

% 书签
\setupinteraction[state=start]
\setupinteractionscreen[option=bookmark]
\placebookmarks[chapter,section,subsection,subsubsection][chapter]

% 参考文献
%\setupbibtex[database=bibliography]
\iffalse
\setuppublications[alternative=num]

\setuppublicationlayout[booklet]{%
   \insertauthors{}{\unskip.}{}%
   \inserttitle{ \bgroup\it }{\/\egroup\unskip.}{}%
   \insertbiburl{ URL: }{\unskip.}{}%
   \insertnote{ }{\unskip.}{}}

\setuppublicationlayout[manual]{%
   \insertauthors{}{\unskip.}{}%
   \inserttitle{ \bgroup\it }{\/\egroup\unskip.}{}%
   \insertbiburl{ URL: }{\unskip.}{}%
   \insertnote{ }{\unskip.}{}}
\fi


%---- 其它
\setuptolerance 
  [verytolerant,stretch]

\setupinterlinespace
  [big]

\setupindenting
  [always,first,2em]

\setupfloats
  [indentnext=yes] 

\setupcaptions
  [style=\tfx, headstyle=\normal]

\definesymbol[1][\dag]
\setupitemize[each][packed,serried,inmargin][symbol=1,margin=2em]

\setupfootnotes[way=bypage]

\setupinmargin[left,right][style=\tfx]

\definedescription[definition]
        [location=top,hang=20,width=broad,indenting=always,style=\ss,headstyle=\bf]

\def\mycite#1{\high{\cite[#1]}}

\definesystemvariable{thu}    % general namespace
\definesystemvariable{thucn}  % for chinese namespace
\definesystemvariable{thuen}  % for english namespace

\startvariables all
	title: 		title
	degree:		degree
	department: 	department
	discipline:	discipline
	name:		name
	supervisor:	supervisor
	assupervisor:	assupervisor	% associated supervisor
	cosupervisor:	cosupervisor
	date:		date
\stopvariables

\setuplabeltext 
	[title=题目,
	 department=培养单位,
	 discipline=学科,
	 name=研究生,
	 supervisor=指导教师,
	 assupervisor=副指导教师,
	 cosupervisor=联合导师]

%D \type{\setupcover} collects meta-data to creat the cover. Since doctor dissertation 
%D has both Chinese and English covers, so we need collect two sets data, which is distinguished
%D with the first argument \type{cn | en}, and the second argument provides setup details. If only
%D one argument is supplied, we take \type{cn} as default.
\def\setupcover{%
  \dodoubleempty\dosetupcover}
\def\dosetupcover[#1][#2]{%
  \ifsecondargument
    \getparameters[\??thu#1][#2]%
  \else
    \getparameters[\??thucn][#1]%
  \fi}

\def\putentry#1{%
  \@EA\doifsomething\@EA{\csname @@thucn#1\endcsname}{%
    \labeltext{#1}:\getvalue{@@thucn#1}\par}}

\def\makecncover{%
  \startstandardmakeup
  % \setuplayout []
  \setupindenting[\v!no]
  \setupinterlinespace[\v!big]	
  \setupalign[\v!middle] % or: \raggedcenter, \setupalignment[]\stopalignment
    {\bfd\ss\@@thucntitle}
    \blank[\v!small]
    (申请清华大学 \@@thucndegree 学位论文)
    \blank[2*\v!big]

    \putentry{department}
    \putentry{discipline}
    \putentry{name}
    \putentry{assupervisor}
    \putentry{cosupervisor}

    \doifdefinedelse{@@thucndate}%
      {\doifemptyelse{\@@thucndate}{\currentdate}{\@@thucndate}}%
      {\currentdate} % todo: replace \currentdate with chinese date
    \vfill	
  \stopstandardmakeup}

\def\makeencover{%
  \startstandardmakeup
  % \setuplayout []
  \setupindenting[\v!no]
  \setupinterlinespace[\v!big]
  \setupalign[\v!middle]  
    {\bfd\ss\@@thuentitle}
    \blank[3*\v!big]
    \startlines
      Dissertation Submitted to
      {\bf Tsinghua University}
      in partial fulfillment of the requirement
      for the degree of
      {\bf\ss\@@thuendegree}
      \blank[\v!big]
      by
     {\bf\ss\@@thuenname\par\@@thuendiscipline}
    \stoplines
      \starttabulate [|r|l|]
        \NC Dissertation Supervisor: \NC \@@thuensupervisor \NC \NR
        \NC Associate Supervisor: \NC \@@thuenassupervisor  \NC \NR
      \stoptabulate
    \blank[\v!big]
    \bgroup\bf\ss%
    \doifdefinedelse{@@thuendate}%
      {\doifemptyelse{\@@thuendate}{\currentdate}{\@@thuendate}}%
      {\currentdate} 
    \egroup
   \vfill
  \stopstandardmakeup}

\startbuffer [authorization]
本人完全了解清华大学有关保留、使用学位论文的规定,即:

清华大学拥有在著作权法规定范围内学位论文的使用权,其中包括:(1)已获学位的研究生
必须按学校规定提交学位论文,学校可以采用影印、缩印或其他复制手段保存研究生上交的
学位论文;(2)为教学和科研目的,学校可以将公开的学位论文作为资料在图书馆、资料
室等场所供校内师生阅读,或在校园网上供校内师生浏览部分内容;(3)根据《中华人民
共和国学位条例暂行实施办法》,向国家图书馆报送可以公开的学位论文。

本人保证遵守上述规定。

{\bf(保密的论文在解密后应遵守此规定)}
\stopbuffer

\def\makeauthorization{%
  \startstandardmakeup %[doublesided=no]
  \title{关于学位论文使用授权的说明}

  \setupindenting[\v!yes,2em]
  \setupinterlinespace[\v!auto,\v!big]
  \getbuffer[authorization]

  \startalignment[flushleft]
    \def\@sigline##1{\hbox{##1}:\underbar{\hbox to 5em{}}}
    \@sigline{作者签名} \@sigline{导师签名}
  
    \@sigline{日期} \@sigline{日期}
  \stopalignment
  \vfill
  \stopstandardmakeup}

\startsetups [cover]
  \makecncover
  \doifmode{doctor}{\makeencover}
  \makeauthorization
\stopsetups  

\def\startabstract{%
  \dosingleargument\dostartabstract}
  
\def\dostartabstract[#1]{%
  \dostartbuffer[abstract-#1][startabstract][stopabstract]}

\def\keywords{%
  \dodoubleempty\dokeywords}

\def\dokeywords[#1][#2]{%
  \ifsecondargument
    \dodokeywords[#1][#2]
  \else
    \dodokeywords[cn][#2]
  \fi}

\def\dodokeywords[#1][#2]{%
  \@EA\gdef\csname \??thu#1keywords\endcsname{#2}}
  
\startsetups [abstract]
  \title{摘要}
  \getbuffer[abstract-cn]\par
  
  \noindent{\bf 关键词:} \doifdefined{@@thucnkeywords}{\@@thucnkeywords}
  
  \title{Abstract}
  \getbuffer[abstract-en]\par
  
  \noindent{\bf Keywords:} \doifdefined{@@thuenkeywords}{\@@thuenkeywords}
\stopsetups
  
\stopmodule
\protect \endinput

% test font availability
% \doiffontpresentelse{A} {
	% \definefontsynonym[sans][A]
% } {
	% \definefontsynonym[sans][B]
% }

% i guess that this also works

% \expanded{\definefontsynonym[sans][\doiffontpresentelse{A}{A}{B}]}

%\showframe
% http://wiki.contextgarden.net/Document_Titles
% http://source.contextgarden.net/doc/context/manuals/allkind/mcommon.tex
