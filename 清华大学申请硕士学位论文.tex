% !TeX encoding = UTF-8
% !TeX program = xelatex
% !TeX spellcheck = en_US

\documentclass[degree=master, degree-type=professional]{thuthesis}
  % 学位 degree:
  %   doctor | master | bachelor | postdoc
  % 学位类型 degree-type:
  %   academic(默认)| professional
  % 语言 language
  %   chinese(默认)| english
  % 字体库 fontset
  %   windows | mac | fandol | ubuntu
  % 建议终版使用 Windows 平台的字体编译


% 论文基本配置,加载宏包等全局配置
% !TeX root = ./thuthesis-example.tex

% 论文基本信息配置

\thusetup{
  %******************************
  % 注意:
  %   1. 配置里面不要出现空行
  %   2. 不需要的配置信息可以删除
  %   3. 建议先阅读文档中所有关于选项的说明
  %******************************
  %
  % 输出格式
  %   选择打印版(print)或用于提交的电子版(electronic),前者会插入空白页以便直接双面打印
  %
  output = print,
  % 格式类型
  %   默认为论文(thesis),也可以设置为开题报告(proposal)
  % thesis-type = proposal,
  %
  % 标题
  %   可使用“\\”命令手动控制换行
  %
  title  = {清华大学学位论文 \LaTeX{} 模板\\使用示例文档 v\version},
  title* = {An Introduction to \LaTeX{} Thesis Template of Tsinghua
            University v\version},
  %
  % 学科门类
  %   1. 学术型
  %      - 中文
  %        需注明所属的学科门类,例如:
  %        哲学、经济学、法学、教育学、文学、历史学、理学、工学、农学、医学、
  %        军事学、管理学、艺术学
  %      - 英文
  %        博士:Doctor of Philosophy
  %        硕士:
  %          哲学、文学、历史学、法学、教育学、艺术学门类,公共管理学科
  %          填写“Master of Arts“,其它填写“Master of Science”
  %   2. 专业型
  %      直接填写专业学位的名称,例如:
  %      教育博士、工程硕士等
  %      Doctor of Education, Master of Engineering
  %   3. 本科生不需要填写
  %
  degree-category  = {工学硕士},
  degree-category* = {Master of Science},
  %
  % 培养单位
  %   填写所属院系的全名
  %
  department = {计算机科学与技术系},
  %
  % 学科
  %   1. 研究生学术型学位,获得一级学科授权的学科填写一级学科名称,其他填写二级学科名称
  %   2. 本科生填写专业名称,第二学位论文需标注“(第二学位)”
  %
  discipline  = {计算机科学与技术},
  discipline* = {Computer Science and Technology},
  %
  % 专业领域
  %   1. 设置专业领域的专业学位类别,填写相应专业领域名称
  %   2. 2019 级及之前工程硕士学位论文,在 `engineering-field` 填写相应工程领域名称
  %   3. 其他专业学位类别的学位论文无需此信息
  %
  % professional-field  = {计算机技术},
  % professional-field* = {Computer Technology},
  %
  % 姓名
  %
  author  = {薛瑞尼},
  author* = {Xue Ruini},
  %
  % 学号
  % 仅当书写开题报告时需要(同时设置 `thesis-type = proposal')
  %
  % student-id = {2000310000},
  %
  % 指导教师
  %   中文姓名和职称之间以英文逗号“,”分开,下同
  %
  supervisor  = {郑纬民, 教授},
  supervisor* = {Professor Zheng Weimin},
  %
  % 副指导教师
  %
  associate-supervisor  = {陈文光, 教授},
  associate-supervisor* = {Professor Chen Wenguang},
  %
  % 联合指导教师
  %
  % co-supervisor  = {某某某, 教授},
  % co-supervisor* = {Professor Mou Moumou},
  %
  % 日期
  %   使用 ISO 格式;默认为当前时间
  %
  % date = {2019-07-07},
  %
  % 是否在中文封面后的空白页生成书脊(默认 false)
  %
  include-spine = false,
  %
  % 密级和年限
  %   秘密, 机密, 绝密
  %
  % secret-level = {秘密},
  % secret-year  = {10},
  %
  % 博士后专有部分
  %
  % clc                = {分类号},
  % udc                = {UDC},
  % id                 = {编号},
  % discipline-level-1 = {计算机科学与技术},  % 流动站(一级学科)名称
  % discipline-level-2 = {系统结构},          % 专业(二级学科)名称
  % start-date         = {2011-07-01},        % 研究工作起始时间
}

% 载入所需的宏包

% 定理类环境宏包
\usepackage{amsthm}
% 也可以使用 ntheorem
% \usepackage[amsmath,thmmarks,hyperref]{ntheorem}

\thusetup{
  %
  % 数学字体
  % math-style = GB,  % GB | ISO | TeX
  math-font  = xits,  % stix | xits | libertinus
}

% 可以使用 nomencl 生成符号和缩略语说明
% \usepackage{nomencl}
% \makenomenclature

% 表格加脚注
\usepackage{threeparttable}

% 表格中支持跨行
\usepackage{multirow}

% 固定宽度的表格。
% \usepackage{tabularx}

% 跨页表格
\usepackage{longtable}

% 算法
\usepackage{algorithm}
\usepackage{algorithmic}

% 量和单位
\usepackage{siunitx}

% 参考文献使用 BibTeX + natbib 宏包
% 顺序编码制
\usepackage[sort]{natbib}
\bibliographystyle{thuthesis-numeric}

% 著者-出版年制
% \usepackage{natbib}
% \bibliographystyle{thuthesis-author-year}

% 生命科学学院要求使用 Cell 参考文献格式(2023 年以前使用 author-date 格式)
% \usepackage{natbib}
% \bibliographystyle{cell}

% 本科生参考文献的著录格式
% \usepackage[sort]{natbib}
% \bibliographystyle{thuthesis-bachelor}

% 参考文献使用 BibLaTeX 宏包
% \usepackage[style=thuthesis-numeric]{biblatex}
% \usepackage[style=thuthesis-author-year]{biblatex}
% \usepackage[style=gb7714-2015]{biblatex}
% \usepackage[style=apa]{biblatex}
% \usepackage[style=mla-new]{biblatex}
% \usepackage[notes,backend=biber]{biblatex-chicago}
% 声明 BibLaTeX 的数据库
% \addbibresource{ref/refs.bib}

% 参考文献使用 citation-style-language 宏包(试验性)
% 需要 TeX Live 2023+ 以及最新版 citation-style-language 宏包
% \usepackage{citation-style-language}
% \cslsetup{style = thuthesis-numeric}
% \cslsetup{style = thuthesis-author-year}
% \cslsetup{style = apa}
% \cslsetup{style = modern-language-association}
% \cslsetup{style = cell}  % 生物医学
% \cslsetup{style = tsinghua-university-academy-of-arts-and-design}  % 美术学院
% 声明 CSL 的数据库
% \addbibresource{ref/refs.bib}

% 定义所有的图片文件在 figures 子目录下
\graphicspath{{figures/}}

% 数学命令
\makeatletter
\newcommand\dif{%  % 微分符号
  \mathop{}\!%
  \ifthu@math@style@TeX
    d%
  \else
    \mathrm{d}%
  \fi
}
\makeatother

% hyperref 宏包在最后调用
\usepackage{hyperref}



\begin{document}

% 封面
\maketitle

% 学位论文指导小组、公开评阅人和答辩委员会名单
% 本科生不需要
% % !TeX root = ../thuthesis-example.tex

\begin{committee}[name={毕业论文公开评阅人和答辩委员会名单}]

  \newcolumntype{C}[1]{@{}>{\centering\arraybackslash}p{#1}}

  \section*{公开评阅人名单}

  \begin{center}
    \begin{tabular}{C{3cm}C{3cm}C{9cm}@{}}
      刘XX & 教授   & 清华大学                    \\
      陈XX & 副教授 & XXXX大学                    \\
      杨XX & 研究员 & 中国XXXX科学院XXXXXXX研究所 \\
    \end{tabular}
  \end{center}


  \section*{答辩委员会名单}

  \begin{center}
    \begin{tabular}{C{2.75cm}C{2.98cm}C{4.63cm}C{4.63cm}@{}}
      主席 & 赵XX                  & 教授                    & 清华大学       \\
      委员 & 刘XX                  & 教授                    & 清华大学       \\
          & \multirow{2}{*}{杨XX} & \multirow{2}{*}{研究员} & 中国XXXX科学院 \\
          &                       &                         & XXXXXXX研究所  \\
          & 黄XX                  & 教授                    & XXXX大学       \\
          & 周XX                  & 副教授                  & XXXX大学       \\
      秘书 & 吴XX                  & 助理研究员              & 清华大学       \\
    \end{tabular}
  \end{center}

\end{committee}



% 也可以导入 Word 版转的 PDF 文件
% \begin{committee}[file=figures/committee.pdf]
% \end{committee}


% 使用授权的说明
% 本科生开题报告不需要
% \copyrightpage
% 将签字扫描后授权文件 scan-copyright.pdf 替换原始页面
% \copyrightpage[file=scan-copyright.pdf]

% \frontmatter
% % !TeX root = ../thuthesis-example.tex

% 中英文摘要和关键字

\begin{abstract}
  论文的摘要是对论文研究内容和成果的高度概括。
  摘要应对论文所研究的问题及其研究目的进行描述,对研究方法和过程进行简单介绍,对研究成果和所得结论进行概括。
  摘要应具有独立性和自明性,其内容应包含与论文全文同等量的主要信息。
  使读者即使不阅读全文,通过摘要就能了解论文的总体内容和主要成果。

  论文摘要的书写应力求精确、简明。
  切忌写成对论文书写内容进行提要的形式,尤其要避免“第 1 章……;第 2 章……;……”这种或类似的陈述方式。

  关键词是为了文献标引工作、用以表示全文主要内容信息的单词或术语。
  关键词不超过 5 个,每个关键词中间用分号分隔。

  % 关键词用“英文逗号”分隔,输出时会自动处理为正确的分隔符
  \thusetup{
    keywords = {关键词 1, 关键词 2, 关键词 3, 关键词 4, 关键词 5},
  }
\end{abstract}

\begin{abstract*}
  An abstract of a dissertation is a summary and extraction of research work and contributions.
  Included in an abstract should be description of research topic and research objective, brief introduction to methodology and research process, and summary of conclusion and contributions of the research.
  An abstract should be characterized by independence and clarity and carry identical information with the dissertation.
  It should be such that the general idea and major contributions of the dissertation are conveyed without reading the dissertation.

  An abstract should be concise and to the point.
  It is a misunderstanding to make an abstract an outline of the dissertation and words “the first chapter”, “the second chapter” and the like should be avoided in the abstract.

  Keywords are terms used in a dissertation for indexing, reflecting core information of the dissertation.
  An abstract may contain a maximum of 5 keywords, with semi-colons used in between to separate one another.

  % Use comma as separator when inputting
  \thusetup{
    keywords* = {keyword 1, keyword 2, keyword 3, keyword 4, keyword 5},
  }
\end{abstract*}


% 目录
\tableofcontents

% 插图和附表清单
% 本科生的插图索引和表格索引需要移至正文之后、参考文献前
% \listoffiguresandtables  % 插图和附表清单(仅限研究生)
% \listoffigures           % 插图清单
% \listoftables            % 附表清单

% 符号对照表
% \input{data/denotation}


% 正文部分
\mainmatter
\chapter{项目背景}
% \label{cha:intro}
\label{cha:background}


\section{社会背景}
法律是治国之重器,法律表明了国家权力的运行和国家意志的体现。法律在规范个人、保障社会秩序方面起到了非常重要的作用,同时法律也是我们每一个人维护自身权益的保证。因此,法律对于国家、社会、个人都是相当重要的。自新中国成立以来,全国上下各个司法机构、学术领域产生了大量的法律文书、法学期刊等文件。这些法学文件一直以来被法学领域的学者、法官、律师重视,成为了他们工作中非常重要的一个部分。自国家全面推行依法治国以来,法律在我们日常生活中扮演着越来越重要的角色。但是,当人们在生活中遇到法律问题时,却常常因为缺乏专业知识而不知道如何进行维权。同时,法学领域的学者、工作人员在面对浩如烟海的法律文本时,也常常因为缺乏好的检索工具和阅读系统,而需要花费大量的时间来查找相关材料,大大降低了工作效率。

据统计,近几年来,中国每天发生的法律案件日益增加,2018年人民法院受理的案件高达2800万件,创下历史新高 ;全国共有12万法官,平均每个法官一年审理案件233件;全国共拥有36.5万的职业律师,一年代理的诉讼案件达到465万件,却不足一年案件总数的20\%。上述数据显示,法律服务在中国有着庞大的市场规模。但是由于法律人才的供应不足,市场上对法律服务的需求始终没有办法得到满足。 

在另一方面,技术的创新给我们的生活带来了许许多多的改变,其中表现最突出的便是人工智能与大数据的技术。伴随着深度学习的发展,人工智能中的许多领域在近几年来取得了很大的进步。自然语言处理技术便是其中的代表,在该领域中文本分类、知识图谱、信息抽取、机器翻译等任务上,现有模型纷纷打破了前人的种种记录。技术的创新带来的是生活的改变,而这样的改变也逐渐渗透到法学领域。2018年,司法部提出“加快‘数字法治、智慧司法’建设”的口号 ,这也让法学学者也越发注重人工智能在司法领域的应用。


\section{商业背景}
随着法律智能受到社会的广泛关注,各大有关公司也都推出了相应的产品。其中最具代表性的法律工具网站有:中国裁判文书网、北大法宝、法信网、法咚咚、搜狗法律等。这些产品大多以基于词匹配的案件文书检索为核心。受限于统计方法的检索能力,这些产品大多面向法律专业人士,旨在为他们提供基础的法律服务。


在这样的契机之下,团队构建了一个案情分析平台——JudgeAI,尝试将自然语言技术运用到法学领域的各个问题之上,旨在利用自然语言处理的技术的方法来解决实际法律问题,为大家的工作、生活提供便利。我们实现了对案情的全方面多角度的分析,在输入一段文本形式案情描述之后,我们平台将提供:案件的案由/罪名预测、判案要素预测、相关法条预测、相似案件检索、关键词提取和相关问题推荐的功能。以此让用户能够尽快捕捉到文本中的有用信息,并作出相应的判断,从而达到辅助判决的作用。相比于已有的产品、算法,我们团队在这些任务上提升了对长文本语义的理解,超过以往模型,达到了更高的预测精度。



% !TeX root = ../thuthesis-example.tex

\chapter{图表示例}

\section{插图}

图片通常在 \env{figure} 环境中使用 \cs{includegraphics} 插入,如图~\ref{fig:example} 的源代码。
建议矢量图片使用 PDF 格式,比如数据可视化的绘图;
照片应使用 JPG 格式;
其他的栅格图应使用无损的 PNG 格式。
注意,LaTeX 不支持 TIFF 格式;EPS 格式已经过时。

\begin{figure}
  \centering
  \includegraphics[width=0.5\linewidth]{example-image-a.pdf}
  \caption*{国外的期刊习惯将图表的标题和说明文字写成一段,需要改写为标题只含图表的名称,其他说明文字以注释方式写在图表下方,或者写在正文中。}
  \caption{示例图片标题}
  \label{fig:example}
\end{figure}

若图或表中有附注,采用英文小写字母顺序编号,附注写在图或表的下方。
国外的期刊习惯将图表的标题和说明文字写成一段,需要改写为标题只含图表的名称,其他说明文字以注释方式写在图表下方,或者写在正文中。

如果一个图由两个或两个以上分图组成时,各分图分别以 (a)、(b)、(c)...... 作为图序,并须有分图题。
推荐使用 \pkg{subcaption} 宏包来处理, 比如图~\ref{fig:subfig-a} 和图~\ref{fig:subfig-b}。

\begin{figure}
  \centering
  \subcaptionbox{分图 A\label{fig:subfig-a}}
    {\includegraphics[width=0.35\linewidth]{example-image-a.pdf}}
  \subcaptionbox{分图 B\label{fig:subfig-b}}
    {\includegraphics[width=0.35\linewidth]{example-image-b.pdf}}
  \caption{多个分图的示例}
  \label{fig:multi-image}
\end{figure}



\section{表格}

表应具有自明性。为使表格简洁易读,尽可能采用三线表,如表~\ref{tab:three-line}。
三条线可以使用 \pkg{booktabs} 宏包提供的命令生成。

\begin{table}
  \centering
  \caption{三线表示例}
  \begin{tabular}{ll}
    \toprule
    文件名          & 描述                         \\
    \midrule
    thuthesis.dtx   & 模板的源文件,包括文档和注释 \\
    thuthesis.cls   & 模板文件                     \\
    thuthesis-*.bst & BibTeX 参考文献表样式文件    \\
    \bottomrule
  \end{tabular}
  \label{tab:three-line}
\end{table}

表格如果有附注,尤其是需要在表格中进行标注时,可以使用 \pkg{threeparttable} 宏包。
研究生要求使用英文小写字母 a、b、c……顺序编号,本科生使用圈码 ①、②、③……编号。

\begin{table}
  \centering
  \begin{threeparttable}[c]
    \caption{带附注的表格示例}
    \label{tab:three-part-table}
    \begin{tabular}{ll}
      \toprule
      文件名                 & 描述                         \\
      \midrule
      thuthesis.dtx\tnote{a} & 模板的源文件,包括文档和注释 \\
      thuthesis.cls\tnote{b} & 模板文件                     \\
      thuthesis-*.bst        & BibTeX 参考文献表样式文件    \\
      \bottomrule
    \end{tabular}
    \begin{tablenotes}
      \item [a] 可以通过 xelatex 编译生成模板的使用说明文档;
        使用 xetex 编译 \file{thuthesis.ins} 时则会从 \file{.dtx} 中去除掉文档和注释,得到精简的 \file{.cls} 文件。
      \item [b] 更新模板时,一定要记得编译生成 \file{.cls} 文件,否则编译论文时载入的依然是旧版的模板。
    \end{tablenotes}
  \end{threeparttable}
\end{table}

如某个表需要转页接排,可以使用 \pkg{longtable} 宏包,需要在随后的各页上重复表的编号。编号后跟表题(可省略)和“(续)”,置于表上方。续表均应重复表头。

\begin{longtable}{cccc}
    \caption{跨页长表格的表题}
    \label{tab:longtable} \\
    \toprule
    表头 1 & 表头 2 & 表头 3 & 表头 4 \\
    \midrule
  \endfirsthead
    \caption*{续表~\thetable\quad 跨页长表格的表题} \\
    \toprule
    表头 1 & 表头 2 & 表头 3 & 表头 4 \\
    \midrule
  \endhead
    \bottomrule
  \endfoot
  Row 1  & & & \\
  Row 2  & & & \\
  Row 3  & & & \\
  Row 4  & & & \\
  Row 5  & & & \\
  Row 6  & & & \\
  Row 7  & & & \\
  Row 8  & & & \\
  Row 9  & & & \\
  Row 10 & & & \\
\end{longtable}

如果表格的内容长度超过了页面长度,可以使用 \pkg{tabularx} 宏包实现表格内容的自动换行,自动换行的列对齐方式使用X表示。

\begin{table}
  \centering
  \caption{表格内容自动换行示例}
  \begin{tabularx}{\textwidth}{lX}
    \toprule
    文件名          & 描述                         \\
    \midrule
    thuthesis.dtx   & 模板的源文件,包括文档和注释。模板的源文件,包括文档和注释。模板的源文件,包括文档和注释。模板的源文件,包括文档和注释。模板的源文件,包括文档和注释。模板的源文件,包括文档和注释。 \\
    thuthesis.cls   & 模板文件                     \\
    thuthesis-*.bst & BibTeX 参考文献表样式文件    \\
    \bottomrule
  \end{tabularx}
  \label{tab:tab-auto-wrap}
\end{table}

\section{算法}

算法环境可以使用 \pkg{algorithms} 或者 \pkg{algorithm2e} 宏包。

\renewcommand{\algorithmicrequire}{\textbf{输入:}\unskip}
\renewcommand{\algorithmicensure}{\textbf{输出:}\unskip}

\begin{algorithm}
  \caption{Calculate $y = x^n$}
  \label{alg1}
  \small
  \begin{algorithmic}
    \REQUIRE $n \geq 0$
    \ENSURE $y = x^n$

    \STATE $y \leftarrow 1$
    \STATE $X \leftarrow x$
    \STATE $N \leftarrow n$

    \WHILE{$N \neq 0$}
      \IF{$N$ is even}
        \STATE $X \leftarrow X \times X$
        \STATE $N \leftarrow N / 2$
      \ELSE[$N$ is odd]
        \STATE $y \leftarrow y \times X$
        \STATE $N \leftarrow N - 1$
      \ENDIF
    \ENDWHILE
  \end{algorithmic}
\end{algorithm}

\chapter{项目创新点与意义}
\label{cha: significance}

\section{项目创新点}

在学术研究上,团队尝试应用自然语言处理技术解决实际生活中的法学问题。通过长时间的探索与交流,我们总结并提出了多个常见的法学任务,并将其形式化为能够利用自然语言处理模型进行处理的常见任务。并且在多个任务上,团队提出了创新性的模型,达到了不错的效果。

另一方面,通过与法律从业人员长时间的沟通交流,团队提出的任务都是建立在切实解决实际问题的基础之上的。我们致力于打造一个功能全面、技术可靠的案情分析平台。相比于市面上现有的一些法律产品,我们平台的功能更加全面、实用,结果更加可靠。

总结而言,项目有以下亮点与创新:

1)	尝试利用自然语言处理的最前沿的技术解决相关的法律问题。通过对案由预测、相关法条预测、刑期预测、关键词抽取、类案推荐等功能的实现,平台可以帮助法律领域从业人员减免重复工作,成为辅助其工作的好工具;同时也可以为大家身边遇到的法律问题提供的解决方案,成为无法学背景的非专业人士的好帮手。
2)	提出了全面的法学基础任务,能够对每一段案情进行多角度全方位的分析,更好的满足了用户需求。通过与从业人员的多次的深入交流,我们获知了他们在工作中碰到最多的几大问题,并进行针对性的解决,真正做到了充分了解用户需求。
3)	在判决预测模块,我们提出了一个能够捕捉子任务间依赖关系的多任务学习模型,超过了以往模型,实现了state-of-the-art效果。在我们提出的几个任务之间,往往有着很强的依赖关系,例如案由与法条之间具有很强的映射关系。模型通过捕捉这些子任务之间的映射关系,提升其效果。
4)	首次在关键词抽取任务上运用并改进Lattice-LSTM模型。克服了词级别模型过度依赖于中文分词效果、字级别模型语义信息不足的缺点,在传统的序列标注模型上实现了大幅提升。
5)	在类案检索模块,我们提出了一个基于关键词抽取、案件语义理解的模型,做到了在语义层面上的相似性检索。模型首先抽取案情关键词标签,通过标签缩小候选的相似文本集合,再进一步通过不同文章的文章向量之间的距离来衡量案情的相似程度。相比于传统搜索引擎的文本相似性检索,此搜索模块可以做到真正的语义相似性检索。

\section{项目意义}





\chapter{算法实现}
本章,我们将从判决预测、关键词抽取、类案搜索等模块来详细介绍项目中用到的算法模型。在模型中,我们统一使用了THULAC[9]进行中文分词,我们使用的词向量是利用FastText[10]模型,在大规模的法律文书语料库上进行的预训练,我们采用的词向量维度为200维。

对于每一个模块,我们将分别从任务的描述定义、算法的创新点、算法模型细节三个方面来进行详细阐述。


\section{判决预测}
\subsection{任务描述}
在该模块中,我们主要实现了案件的罪名及案由预测、相关法条推荐、刑事案件刑期预测、相关判案要素预测等功能。

输入一段文本形式的案情描述,模型通过阅读文本获取文本中蕴含的语义信息,将案情映射到一个高维向量空间中,得到一个包含有文本信息的文章向量,再根据不同的任务,将文章向量喂给不同的输出层获得不同的分类结果。

\subsection{创新点}
我们将上述所有功能归纳为文本分类问题,通过观察,上述任务之间有很强的前后依赖关系,例如,相关法条推荐结果往往与罪名及案由预测结果高度相关。并且,根据法学领域从业人员介绍,在实际判案时,大家首先会判断案情中的相关要素,并根据要素判断结果来决定最后的案件的判决。因此,我们有以下创新来提升各个任务的效果:

1)	在多任务学习(multi-task learning)中提出了一个新的架构,充分利用各个任务之间的依赖关系来共同提升多个任务的预测效果。

2)	模拟法官判案过程,提出判断案件判案要素作为中间任务,来大幅提升相关任务的效果。


\subsection{算法模型}
算法使用了LSTM作为模型的编码器,使用了全连接神经网络作为输出层。利用了多任务学习的框架,将多个任务进行联合训练得到了效果的提升。

我们在获得案情文本后,可将算法分为以下几步:

1)	输入层:文本分词与词向量映射。通过上文提到的THULAC与FastText完成本步,将文章映射成词向量序列。

2)	编码层:我们利用了LSTM循环神经网络对词向量序列进行编码,得到一个隐向量序列。之后,我们利用了self-attention机制,来获得对应于不同任务的文本信息。

3)	预测层:我们利用了LSTM Cell来捕获不同任务之间的依赖关系,将不同的任务获得的文本向量当做一个序列,使得任务之间可以获取相应被依赖任务的信息,从而提升相应的效果。

4)	输出层:最后,我们应用了全连接神经网络,将包含任务信息的向量映射到相应的分类中。



\section{关键词抽取}
\subsection{任务描述}
在该模块中,我们利用了序列标注模型实现了对案情文本的关键词抽取。任务输入仍旧为一段案情描述的文本,任务的输出为文本中的某一个或者某一些词语,这些词语与法律词汇高度相关,是用来检索相关案件、法条的重要依据。

例如,对于句子:“被告对{\color{red}火灾}发生是否存在{\color{red}过错},应否对原告的合理损失予以{\color{red}赔偿}”中标红的即为关键词,对于检索相关法律法规:“关于{\color{red}火灾过错}引发事故责任划分与{\color{red}赔偿}…”有很强的指导意义。

本模块我们采用了对句子级别的预料进行训练的方式,我们整理了10000条待标注数据,通过寻找专业人士进行人工标注的方法,获取了相应的待标注的训练数据。

在使用时,我们将对每一个句子进行关键词抽取,通过算法规则将所有句子的关键词进行合并、筛选,得到最终文章级别的关键词。关键词抽取的结果将对后续的相关案例检索提供重要的帮助。

\subsection{创新点}

关键词抽取是自然语言处理中,非常常见的任务。在早期,大多数人们使用的是基于词频等特征的无监督学习算法,例如TextRank、LDA、TFIDF算法。对于法律这样一个特定领域,关键词抽取的结果往往是一些法言法语,相比于开放领域上的关键词抽取,有着更强的规律性。因此,我们将目前效果最好的命名实体识别算法运用至该任务。同时,由于训练是在句子级别的语料上进行训练,我们设计了筛选算法,将在所有句子上的抽取结果进行合并筛选,得到了一个较好的效果。

\subsection{算法模型}

该算法将目前前沿的命名实体识别算法——Lattice LSTM[11],通过将句子拆分成字级别,对每一个字一个标签,来判断最终关键词的范围。在传统的基于字级别的模型基础上,加入词级别信息,这样克服了字级别信息不足、词级别过度依赖于分词效果的缺陷。

1)	利用n-gram、分词信息等特征为句子中每一个字进行编码,形成字级别特征;

2)	利用好BiLSTM-CRF的框架,通过BiLSTM对字特征进行编码,再通过CRF捕捉序列信息,进行序列标注;

3)	改造字级别的LSTM模型,将字与字之间匹配成的所有的可能词的特征融合到LSTM的信息传递中,得到一个Lattice-word model;

4)	对每个字预测BIE标签,其中B为begin、I为in、E为end;得到最后序列标注的结果。



\section{类案搜索}
\subsection{任务描述}
该模块实现了相关案例推荐的功能。该功能的实现主要基于上述两个模块的模型。该任务模型以案情描述作为输入,通过关键词标签抽取,检索到一批与该案情相似的法律文书,再利用判决预测模块得到的文本向量,对检索到的文书进行重排序,输出一个按照相似度递减的法律文书集合。

\subsection{创新点}
该任务所用算法与传统的搜索引擎算法不同,传统搜索引擎将检索文本与数据库文本进行比对,进而检索出相似度高的文本段。在我们算法中,我们通过抽取出的关键词标签来初步确定相关的文章的集合,再通过判决预测模块中获得的包含文本语义的向量,来对相关集合进行重排序,得到在语义层面相似度最高的文章。

这样的做法改善了传统搜索引擎只关注文本相似度的缺点,通过捕捉语义信息检索以满足用户需求。

\subsection{算法模型}
如上所述,改算法主要分成以下两步:

1)	抽取关键词,利用关键词抽取模块将案情描述中的关键词抽取出来。

2)	利用关键词标签,确定相关文书的集合;将所有关键词与该案情描述关键词一样的案件抽取出来,形成相关文书集合。

3)	相关程度重排序,利用判决预测模块将第2步获得的文书,转化成包含语义信息的文本向量,通过向量的距离来判断文书与输入案情的相关程度,按照相关程度对文书进行重排序。




\chapter{具体实现与细节展示}
\label{cha:algorithm}

\section{判决预测模块}


\section{关键词抽取模块}


\section{类案检索模块}



\chapter{实验结果}
\label{cha:result}

我们针对性的在案由/罪名预测、相关法条预测、刑期预测、关键词抽取任务上,开展了多个不同的实验。实验结果表明,我们提出的模型超过了以往对应的模型,实现了state-of-the-art的效果。

\section{判决预测模块}
\subsection{实验设置}

针对于判决预测模块提出的多任务学习模型,我们在多个相关数据集上进行了相应的实验。我们根据北大法宝、中国裁判文书网、中国司法挑战赛三个数据来源,构建了相应的三个判决预测数据集。对所有文本,利用了THULAC工具进行分词,统一利用FastText训练了200维的词向量。利用规则、人工标注,从各个文书的判决结果中,抽取出了每一篇文书相应的罪名、法条、刑期等标注信息,并进行了人工检错,并将数据集随机按照8:2的比例划分成了训练集与测试集。我们采用了以下文本分类模型做baseline:

1)	TFIDF+SVM:这是文本分类最常用的传统分类算法,我们应用了词频与逆文档频率TFIDF[12]抽取文档特征,并以此为基础采用了支持向量机SVM[13]进行分类。

2)	CNN:我们采用了卷积神经网络[14]作为编码器,让机器在阅读文章后自动抽取出对分类有用的特征,再将其映射到各个罪名、法条、刑期上。为了对比,我们在改编码器上运用了基本的多任务学习机制,来与我们的模型进行对比。

3)	Fact-Law:[15]是由北大团队自然语言处理团队在2017年提出的罪名预测模型。模型通过引入法条描述,利用Attention机制来提升模型预测罪名的效果。

4)	Pipeline:这是一个基本的多任务学习的模型,该模型为不同的子任务训练不同的编码器,通过在输出层融合不同任务的信息,捕获任务之间的依赖关系。

\subsection{实验结果与分析}
由表中数据可知,我们的模型在相关法条推荐、案由/罪名预测、刑期预测三个任务上超过了之前的baseline,取得了很好的效果。多个数据集上的实验体现了我们提出模型的鲁棒性与有效性。

同时,由于考虑了多个任务之间的相互依赖关系,我们的模型解决了预测结果互相矛盾的问题。例如模型在相关法条预测结果为涉及“盗窃罪”的法条,而在案由/罪名预测时却将被告判断为“故意杀人罪”,这样的预测结果矛盾现象在其他模型中非常常见。我们的模型很好的解决了这样一个问题,因此实现了优越的效果。

\section{关键词抽取模块}
\subsection{实验设置}

针对于关键词预测模块,我们统一使用我们构造的人工标注的数据集作为benchmark进行评测。我们利用了以下传统的关键词抽取模型作为模型baseline进行对比。

1)	CRF:条件随机场是一个传统的基于概率统计的序列标注模型。第一步人工抽取特征,模型通过学习基于这些特征的关键词概率分布来进行训练。

2)	BiLSTM+CRF:双向LSTM与CRF的结合,是目前神经网络运用于序列标注的最主要的模型。利用神经网络抽取特征的能力,来避免人工抽取特征的不全面与高代价。


\subsection{实验结果与分析}

从实验结果可见,我们的模型在关键词提取任务上实现了大幅的提升。相比于传统的字级别的序列标注模型与词级别的序列标注模型,我们使用的模型综合了字级别的信息与词级别的信息,以此避免了词级别模型过度依赖分词效果与字级别模型无法捕捉到足够信息的缺点。



\chapter{时间安排}

为确保研究工作的有序开展和顺利完成,结合本研究的内容和难度,制定如下时间安排:

\begin{table}[htbp]
  \resizebox{\textwidth}{!}{
    \centering
    \begin{tabular}{lp{10cm}}
      \hline
      时间 & 主要工作内容 \\
      \hline
      2023年10月 & 
      \begin{itemize}
        \item 开题报告的准备和答辩
        \item 明确研究目标和研究计划
      \end{itemize} \\
      \hline
      2023年11月 -- 2024年1月 & 
      \begin{itemize}
        \item 深入开展文献综述,系统学习复杂系统理论和网络分析方法
        \item 分析新产品导入(NPI)过程的复杂性特征
        \item 初步构建NPI过程的网络模型框架
      \end{itemize} \\
      \hline
      2024年2月 -- 2024年3月 & 
      \begin{itemize}
        \item 完善NPI过程的网络模型
        \item 确定网络分析的指标和方法
        \item 开发或选择适当的网络分析工具
      \end{itemize} \\
      \hline
      2024年4月 -- 2024年5月 & 
      \begin{itemize}
        \item 收集案例研究所需的数据
        \item 对NPI网络模型进行实证分析
        \item 识别关键节点、关键路径和瓶颈问题
      \end{itemize} \\
      \hline
      2024年6月 -- 2024年7月 & 
      \begin{itemize}
        \item 基于分析结果提出NPI过程的优化策略
        \item 与企业相关人员讨论并修正优化方案
        \item 初步验证优化策略的可行性
      \end{itemize} \\
      \hline
      2024年8月 -- 2024年9月 & 
      \begin{itemize}
        \item 完成论文初稿的撰写
        \item 对论文进行修改和完善
        \item 与导师多次沟通,征求修改意见
      \end{itemize} \\
      \hline
      2024年10月 & 
      \begin{itemize}
        \item 论文定稿,准备论文答辩材料
        \item 参加论文预答辩,根据反馈意见进行修改
      \end{itemize} \\
      \hline
      2024年11月 & 
      \begin{itemize}
        \item 完成论文的最终定稿
        \item 提交论文,准备正式答辩
      \end{itemize} \\
      \hline
      2024年12月 & 
      \begin{itemize}
        \item 参加硕士学位论文答辩
        \item 根据答辩委员会意见进行必要的修改
        \item 办理毕业相关手续
      \end{itemize} \\
      \hline
    \end{tabular}
  }
  \label{tab:schedule}
\end{table}


% 其他部分
\backmatter

% 参考文献
\bibliography{ref/refs}  % 参考文献使用 BibTeX 编译
% \printbibliography       % 参考文献使用 BibLaTeX 编译

% 附录
% 本科生需要将附录放到声明之后,个人简历之前
% \appendix
% % !TeX root = ../thuthesis-example.tex

\begin{survey}
\label{cha:survey}

\title{Title of the Survey}
\maketitle


\tableofcontents


本科生的外文资料调研阅读报告。


\section{Figures and Tables}

\subsection{Figures}

An example figure in appendix (Figure~\ref{fig:appendix-survey-figure}).

\begin{figure}
  \centering
  \includegraphics[width=0.6\linewidth]{example-image-a.pdf}
  \caption{Example figure in appendix}
  \label{fig:appendix-survey-figure}
\end{figure}


\subsection{Tables}

An example table in appendix (Table~\ref{tab:appendix-survey-table}).

\begin{table}
  \centering
  \caption{Example table in appendix}
  \begin{tabular}{ll}
    \toprule
    File name       & Description                                         \\
    \midrule
    thuthesis.dtx   & The source file including documentation and comments \\
    thuthesis.cls   & The template file                                   \\
    thuthesis-*.bst & BibTeX styles                                       \\
    thuthesis-*.bbx & BibLaTeX styles for bibliographies                  \\
    thuthesis-*.cbx & BibLaTeX styles for citations                       \\
    \bottomrule
  \end{tabular}
  \label{tab:appendix-survey-table}
\end{table}


\section{Equations}

An example equation in appendix (Equation~\eqref{eq:appendix-survey-equation}).
\begin{equation}
  \frac{1}{2 \uppi \symup{i}} \int_\gamma f = \sum_{k=1}^m n(\gamma; a_k) \mathscr{R}(f; a_k)
  \label{eq:appendix-survey-equation}
\end{equation}


\section{Citations}

Example\cite{dupont1974bone} citations\cite{merkt1995rotational} in appendix
\cite{dupont1974bone,merkt1995rotational}.


% 默认使用正文的参考文献样式;
% 如果使用 BibTeX,可以切换为其他兼容 natbib 的 BibTeX 样式。
\bibliographystyle{unsrtnat}
% \bibliographystyle{IEEEtranN}

% 默认使用正文的参考文献 .bib 数据库;
% 如果使用 BibTeX,可以改为指定数据库,如 \bibliography{ref/refs}。
\printbibliography

\end{survey}
       % 本科生:外文资料的调研阅读报告
% % !TeX root = ../thuthesis-example.tex

\begin{translation}
\label{cha:translation}

\title{书面翻译题目}
\maketitle

\tableofcontents


本科生的外文资料书面翻译。


\section{图表示例}

\subsection{图}

附录中的图片示例(图~\ref{fig:appendix-translation-figure})。

\begin{figure}
  \centering
  \includegraphics[width=0.6\linewidth]{example-image-a.pdf}
  \caption{附录中的图片示例}
  \label{fig:appendix-translation-figure}
\end{figure}


\subsection{表格}

附录中的表格示例(表~\ref{tab:appendix-translation-table})。

\begin{table}
  \centering
  \caption{附录中的表格示例}
  \begin{tabular}{ll}
    \toprule
    文件名          & 描述                         \\
    \midrule
    thuthesis.dtx   & 模板的源文件,包括文档和注释 \\
    thuthesis.cls   & 模板文件                     \\
    thuthesis-*.bst & BibTeX 参考文献表样式文件    \\
    thuthesis-*.bbx & BibLaTeX 参考文献表样式文件  \\
    thuthesis-*.cbx & BibLaTeX 引用样式文件        \\
    \bottomrule
  \end{tabular}
  \label{tab:appendix-translation-table}
\end{table}


\section{数学公式}

附录中的数学公式示例(公式\eqref{eq:appendix-translation-equation})。
\begin{equation}
  \frac{1}{2 \uppi \symup{i}} \int_\gamma f = \sum_{k=1}^m n(\gamma; a_k) \mathscr{R}(f; a_k)
  \label{eq:appendix-translation-equation}
\end{equation}


\section{文献引用}

附录\cite{dupont1974bone}中的参考文献引用\cite{merkt1995rotational}示例
\cite{dupont1974bone,merkt1995rotational}。


\appendix

\section{附录}

附录的内容。


% 书面翻译的参考文献
% 默认使用正文的参考文献样式;
% 如果使用 BibTeX,可以切换为其他兼容 natbib 的 BibTeX 样式。
\bibliographystyle{unsrtnat}
% \bibliographystyle{IEEEtranN}

% 默认使用正文的参考文献 .bib 数据库;
% 如果使用 BibTeX,可以改为指定数据库,如 \bibliography{ref/refs}。
\printbibliography

% 书面翻译对应的原文索引
\begin{translation-index}
  \nocite{mellinger1996laser}
  \nocite{bixon1996dynamics}
  \nocite{carlson1981two}
  \bibliographystyle{unsrtnat}
  \printbibliography
\end{translation-index}

\end{translation}
  % 本科生:外文资料的书面翻译
% % !TeX root = ../thuthesis-example.tex

\chapter{补充内容}

附录是与论文内容密切相关、但编入正文又影响整篇论文编排的条理和逻辑性的资料,例如某些重要的数据表格、计算程序、统计表等,是论文主体的补充内容,可根据需要设置。

附录中的图、表、数学表达式、参考文献等另行编序号,与正文分开,一律用阿拉伯数字编码,
但在数码前冠以附录的序号,例如“图~\ref{fig:appendix-figure}”,
“表~\ref{tab:appendix-table}”,“式\eqref{eq:appendix-equation}”等。


\section{插图}

% 附录中的插图示例(图~\ref{fig:appendix-figure})。

\begin{figure}
  \centering
  \includegraphics[width=0.6\linewidth]{example-image-a.pdf}
  \caption{附录中的图片示例}
  \label{fig:appendix-figure}
\end{figure}


\section{表格}

% 附录中的表格示例(表~\ref{tab:appendix-table})。

\begin{table}
  \centering
  \caption{附录中的表格示例}
  \begin{tabular}{ll}
    \toprule
    文件名          & 描述                         \\
    \midrule
    thuthesis.dtx   & 模板的源文件,包括文档和注释 \\
    thuthesis.cls   & 模板文件                     \\
    thuthesis-*.bst & BibTeX 参考文献表样式文件    \\
    thuthesis-*.bbx & BibLaTeX 参考文献表样式文件  \\
    thuthesis-*.cbx & BibLaTeX 引用样式文件        \\
    \bottomrule
  \end{tabular}
  \label{tab:appendix-table}
\end{table}


\section{数学表达式}

% 附录中的数学表达式示例(式\eqref{eq:appendix-equation})。
\begin{equation}
  \frac{1}{2 \uppi \symup{i}} \int_\gamma f = \sum_{k=1}^m n(\gamma; a_k) \mathscr{R}(f; a_k)
  \label{eq:appendix-equation}
\end{equation}


\section{文献引用}

附录\cite{dupont1974bone}中的参考文献引用\cite{zhengkaiqing1987}示例
\cite{dupont1974bone,zhengkaiqing1987}。

\printbibliography


% 致谢
% \input{data/acknowledgements}

% 声明
% 本科生开题报告不需要
% \statement
% 将签字扫描后的声明文件 scan-statement.pdf 替换原始页面
% \statement[file=scan-statement.pdf]
% 本科生编译生成的声明页默认不加页脚,插入扫描版时再补上;
% 研究生编译生成时有页眉页脚,插入扫描版时不再重复。
% 也可以手动控制是否加页眉页脚
% \statement[page-style=empty]
% \statement[file=scan-statement.pdf, page-style=plain]

% 个人简历、在学期间完成的相关学术成果
% 本科生可以附个人简历,也可以不附个人简历
% % !TeX root = ../thuthesis-example.tex

\begin{resume}

  \section*{个人简历}

  197× 年 ×× 月 ×× 日出生于四川××县。

  1992 年 9 月考入××大学化学系××化学专业,1996 年 7 月本科毕业并获得理学学士学位。

  1996 年 9 月免试进入清华大学化学系攻读××化学博士至今。


  \section*{在学期间完成的相关学术成果}

  \subsection{学术论文}

  \begin{achievements}
    \item Yang Y, Ren T L, Zhang L T, et al. Miniature microphone with silicon-based ferroelectric thin films[J]. Integrated Ferroelectrics, 2003, 52:229-235.
    \item 杨轶, 张宁欣, 任天令, 等. 硅基铁电微声学器件中薄膜残余应力的研究[J]. 中国机械工程, 2005, 16(14):1289-1291.
    \item 杨轶, 张宁欣, 任天令, 等. 集成铁电器件中的关键工艺研究[J]. 仪器仪表学报, 2003, 24(S4):192-193.
    \item Yang Y, Ren T L, Zhu Y P, et al. PMUTs for handwriting recognition. In press[J]. (已被Integrated Ferroelectrics录用)
  \end{achievements}


  \subsection{专利}

  \begin{achievements}
    \item 任天令, 杨轶, 朱一平, 等. 硅基铁电微声学传感器畴极化区域控制和电极连接的方法: 中国, CN1602118A[P]. 2005-03-30.
    \item Ren T L, Yang Y, Zhu Y P, et al. Piezoelectric micro acoustic sensor based on ferroelectric materials: USA, No.11/215, 102[P]. (美国发明专利申请号.)
  \end{achievements}

  \subsection*{3  奖项}

  \begin{achievements}
    \item 任天令, 杨轶, 朱一平, 等. 硅基铁电微声学传感器畴极化区域控制和电极连接的方法: 中国, CN1602118A[P]. 2005-03-30.
    \item Ren T L, Yang Y, Zhu Y P, et al. Piezoelectric micro acoustic sensor based on ferroelectric materials: USA, No.11/215, 102[P]. (美国发明专利申请号.)
  \end{achievements}

\end{resume}


% 指导教师/指导小组评语
% 本科生不需要
% % !TeX root = ../thuthesis-example.tex

% \begin{comments}
\begin{comments}[name = {基础导师学术评语}]
% \begin{comments}[name = {Comments from Thesis Supervisor}]
% \begin{comments}[name = {Comments from Thesis Supervision Committee}]

  论文提出了……

\end{comments}


\begin{comments}[name = {临床导师学术评语}]

  论文提出了……

\end{comments}


% 答辩委员会决议书
% 本科生不需要
% \input{data/resolution}

% 本科生的综合论文训练记录表(扫描版)
% \record{file=scan-record.pdf}

\end{document}
